\documentclass[12pt]{article}
\usepackage{graphicx}
\usepackage{algpseudocodex}
\usepackage{amsmath}
\usepackage{geometry}
\usepackage{hyperref}
\usepackage{authblk}
\usepackage{amsthm}
\usepackage{amsmath}
\usepackage{longtable}
\usepackage{booktabs}
\usepackage{algorithm}
\usepackage{bookmark}

\geometry{a4paper, margin=1in}

\title{\Huge Self-organizing Logic}
\author{
    \large SHINDO Keisuke \\
    \texttt{kshindo999@gmail.com}
}
\date{April 8, 2025}

\begin{document}

\maketitle

\begin{abstract}
Self-organizing logic is a theoretical system for a logic circuit operation simulator that automatically generates sequential circuits based on probability, and is a systematic and generalized machine learning theory. The first feature is to perform bidirectional probability propagation using a Bayesian network. The second feature is to form an accurate logic circuit using two propagation probabilities between the nodes of the network. The third feature is to propagate activation values called assumption vectors to this network and perform comparisons and substitutions within the network. The fourth feature is to partially replicate and modify the network as necessary. By using these elements, this self-organizing logic can generate logic circuits with the most probabilistic accuracy. Furthermore, it is possible to express hierarchical states, state transitions, and even natural language. As a result, it is expected that all software elements can be generated autonomously.
\end{abstract}

\noindent\textbf{Keywords:} machine learning, bidirectional Bayesian network, sequential circuit

\clearpage

\tableofcontents

\clearpage


\section{Components of self-organizing logic}\label{components-of-self-organizing-logic}

\subsection{Bidirectional Logic Operations and Linked
Nodes}\label{bidirectional-logic-operations-and-linked-nodes}

\subsubsection{classical sequential circuit}\label{classical-sequential-circuit}

In ordinary logic algebra, the elements of AND, OR, and NOT logic
operations are held as nodes, and they are joined by links to perform
basic logic operations. In an ordinal circuit, a memory function such as
a flip-flop FF or memory is added to this, and CPUs, GPUs, etc. all
correspond to this ordinal circuit. This sequential circuit can
theoretically produce any output sequence for any input sequence.

\begin{figure}[ht]
  \centering
  \includegraphics[width=0.7\textwidth]{SOLE/DocumentImage/SOL_SequencialCircuit.png}
  \caption{Sequential circuit}
  \label{fig:Sequential circuit}
\end{figure}

\subsubsection{Mapping by backpropagation of logic circuits}\label{mapping-by-backpropagation-of-logic-circuits}

Logical operations act to cut and paste sets into each other, as
expressed in Venn diagrams, etc. However, logical operations cannot be
performed between exclusive subsets that do not overlap, such as X and Y
in the figure below. In contrast, a mapping makes it possible to reach
another set that is exclusive with respect to the input set. To do so,
the sets before and after the mapping are considered as subsets of an
even larger set, the Map node, which encompasses them. The input node X
and the output node Y are equal to the result of ANDing with the domain
and codomain nodes, respectively, for the Map node.

\begin{figure}[ht]
  \centering
  \includegraphics[width=0.7\textwidth]{SOLE/DocumentImage/SOL_VennDiagram.png}
  \caption{Basic venn diagram}
  \label{fig:Basic venn diagram}
\end{figure}

The propagation from the mapping source X to the mapping destination Y
is realized by back propagation to the Map node. In order to define
probability propagation between exclusive nodes, the back propagation
from AND1 to Map enlarges the set by complementing the complement part
of domain. The set is then expanded by the Map node.

This associative link using mapping nodes and back propagation allows
the description of state transitions. Furthermore, this action enables
the same behavior as memory devices such as flip-flops and memories, but
in a more generalized manner. For example, a memory readout can be
viewed as a mapping from a set of different spacetimes.

\begin{figure}[ht]
  \centering
  \includegraphics[width=0.7\textwidth]{SOLE/DocumentImage/SOL_MapbySet.png}
  \caption{Set representation of mapping}
  \label{fig:Set representation of mapping}
\end{figure}

\begin{figure}[ht]
  \centering
  \includegraphics[width=0.5\textwidth]{SOLE/DocumentImage/fig_4.png}
  \caption{Representation of a mapping using backpropagation}
  \label{fig:Representation of a mapping using backpropagation}
\end{figure}

\subsection{Bidirectional binomial stochastic Bayesian network}\label{bidirectional-binomial-stochastic-bayesian-network}

\subsubsection{Probability propagation}\label{probability-propagation}

A propagation probability is assigned to each connection link between
logical nodes. The logical value to be propagated is binary, 1 or 0, and
the propagation probability is expressed as a value between 0 and 1.

The forward direction of the link is composed of a pair of propagation
probabilities \(P^{f11}\),\(P^{f00}\), and the backward propagation
probability is also composed of a pair of \(P^{r11}\),\(P^{r00}\).

In the propagation link from A to B, the probability values
\hspace{0pt}\hspace{0pt}\(P(A)\),\(P(B)\) of A and B are related by the
two propagation probabilities \(P^{f11}\),\(P^{f00}\).

\[P(B)=P(A) P^{f11} +(1-P(A))(1-P^{f00})\]

The NOT operation, which inverts values, makes \(P^{f11}\) and
\(P^{f00}\) 0. Non-NOT propagation makes \(P^{f11}\) and \(P^{f00}\) 1.
Logical operations are also defined as probability calculations, and
combine the input probabilities
\(P^{f11}_{1}\),\(P^{f11}_{2}\),\(P^{f11}_{3}\) of multiple logical
operations. Below is an example of AND.

\[P'=P^{f11}_{1} P^{f11}_{2} P^{f11}_{3}... \]

By combining these trivial probability calculations, the propagation
probability of the entire path of a logical operation between any two
nodes is calculated. Unlike neural networks, no threshold functions such
as sigmoid functions are used.

\subsubsection{Backward propagation}\label{backward-propagation}

This section describes the probability of backward propagation of a
link. For example, suppose the following AND node C is generated.

\[  C=A\cap B \]

Conversely, if node C is activated with probability 1, then nodes A and
B can be activated with propagation probability 1. This is backward
propagation.

\[  C \rightarrow B \]

Backward propagation ensures that propagation between any two nodes is
possible. The probabilities of backward propagation,
\(P_r^{11}\),\(P_r^{00}\), are calculated using Bayes' theorem. Let P(A)
be the observation probability of the link source A and P(B) be the
observation probability of the link destination B, and the following
formula can be used to calculate the probability.

\[ P_{r}^{11}=\frac{P(A) P^{11}}{P(B)}\]

\[ P_{r}^{00}=\frac{(1-P(A)) P^{00}}{1-P(B)}\]

In other words, bidirectional belief propagation incorporates Bayes'
theorem in a natural way. In the case of backward propagation of a
logical operation, if all values \hspace{0pt}\hspace{0pt}are determined,
such as the backward propagation of a value of 1 to an AND node, the
propagation probability of the input links is not determined when the
value of 0 is backward propagated to an AND node, but if only one input
link is selected for propagation, the probability of only the selected
link is determined.

\subsection{Assumption vectors and propagation sets}\label{assumption-vectors-and-propagation-sets}

SOL propagates activation values \hspace{0pt}\hspace{0pt}on the network,
but the propagated activation value does not occupy the entire cross
section of the links it propagates through as a set; rather, it
considers that the set of cross sections is divided by the values
\hspace{0pt}\hspace{0pt}of the multiple origin nodes. This propagation
set is unrelated to the sets it passes through on the network, and is
determined by the combination of the values \hspace{0pt}\hspace{0pt}of
the origin nodes. For this reason, the combination of assumption values
\hspace{0pt}\hspace{0pt}of multiple origin nodes is called an
``assumption vector.'' The size of the propagation set is indicated by
this assumption vector. Below, the assumption vector will be abbreviated
to ``AV'' as necessary.

\begin{figure}[ht]
  \centering
  \includegraphics[width=0.7\textwidth]{SOLE/DocumentImage/SOL_AssumptionVector.png}
  \caption{Assumption vectors}
  \label{fig:assumption_vectors}
\end{figure}

\subsubsection{Synthesis of assumption vectors}\label{synthesis-of-assumption-vectors}

When multiple assumption vectors are combined through logical
operations, the resulting assumption vector is the combination of both
assumption vector elements. With the logical AND operation, assumption
vectors are combined when two vectors with probabilities of 1 are
combined. With the logical OR operation, assumption vectors are combined
when two vectors with probabilities of 0 are combined. With the logical
XOR operation, or when combining vectors with intermediate probabilities
between 0 and 1, assumption vectors are combined unconditionally.

\begin{itemize}
\item
  Assumption vector elements that are different among the assumption
  vectors are synthesized as they are.
\item
  Assumption vector elements that are identical among the assumption
  vectors are aggregated into one.
\item
  Composition with collectively inclusive assumption vector elements
  replaces the propagation collectively smaller elements.
\item
  Composites of collectively exclusive vector elements result in the
  propagation set itself becoming an empty set.
\end{itemize}

\subsubsection{Propagation set comparison using assumption vectors}\label{propagation-set-comparison-using-assumption-vectors}

This assumption vector makes it possible to compare the magnitudes of
multiple active states as a propagation set. This comparison is as a
propagating set that has passed through the mapping and is independent
of the actual set.

\begin{itemize}
\item
  If all the assumption set elements of the assumption vectors are
  identical, then both propagating sets are identical regardless of the
  actual set.
\item
  If only one of the non-identical parts of the assumption set elements
  of the assumption vector is identical, then both propagation sets are
  inclusive.
\item
  If non-identical parts of the assumption set elements of the
  assumption vectors are non-identical on both sides, then both
  propagation sets are unrelated and therefore not subject to
  association.
\item
  If there is no overlap at all between the assumption set elements of
  the assumption vectors, the propagation itself is stopped because the
  propagation set vanishes.
\end{itemize}

By managing the strict propagation set using the assumption vector, it
is possible to form an accurate associative link between two nodes. All
link formation in SOL is selected and executed according to this
assumption vector.

\subsubsection{Mapping backpropagation and propagation set
expansion}\label{mapping-backpropagation-and-propagation-set-expansion}

The assumption vectors indicate the inclusion relationship of the sets.
For various combinations of the assumptions in the assumption vector, a
propagation probability is defined for each. This allows the probability
of an event occurring to be calculated for each combination of
assumptions.

\begin{figure}[ht]
  \centering
  \includegraphics[width=0.7\textwidth]{SOLE/DocumentImage/SOL_AssumptionSetsandProbabilities.png}
  \caption{Assumption sets and propagation probabilities}
  \label{fig:assumption_sets_and_propagation_probabilities}
\end{figure}

\subsubsection{Map backpropagation and propagation set
expansion}\label{map-backpropagation-and-propagation-set-expansion}

For nodes X and Y that indicate Boolean values, a method is described
for propagating the activation value to node map by using the action of
expanding the set when backpropagating from node Y to node map.

\begin{figure}[ht]
  \centering
  \includegraphics[width=0.7\textwidth]{SOLE/DocumentImage/SOL_BackwardLogic.png}
  \caption{Back propagation}
  \label{fig:back_propagation}
\end{figure}

As shown in the diagram above, when backpropagating from output Y to
input X for AND, even if the values \hspace{0pt}\hspace{0pt}of all nodes
other than the map node to which backpropagation is directed are
determined, the value of the backpropagation result is not determined.
Specifically, if the input value of AND is 0 and the output value of AND
is 0, the value of the map node backpropagated from AND can be either 0
or 1, with no contradiction. Therefore, backpropagation has room to
expand the propagation set.

\begin{figure}[ht]
  \centering
  \includegraphics[width=0.5\textwidth]{SOLE/DocumentImage/Fig_8.png}
  \caption{Map link propagation using assumption vectors}
  \label{fig:map_link_propagation_using_assumption_vectors}
\end{figure}

Perform back propagation from X to map. When back propagation from AND1
to map, if there is another input of the set domain, it is possible to
expand the complement of the domain. Assume that AV1 propagated from X
and AV2 propagated from domain are a pair of two propagation sets with
complementary values. In this case, if the propagation set of X matches
or is completely contained in the propagation set of the domain, the set
from X is expanded and propagated to map. Expanding the propagation set
means that the elements of the complementary assumption vector AV of the
domain are combined to expand the propagation set, which also means that
the assumption vector AV has one less element.

Since the propagation set is expanded as a result of combining the
hypothesis vectors, it is highly likely that the probability is
uniformly 1 or 0. This is because the propagation set combined by map is
highly likely to be the same as the propagation set before adding the
hypothesis element to AV1, and therefore the probability is also
estimated to be a uniform value, just like the propagation set before
adding the hypothesis element.

In this way, the expanded propagation set is propagated from map to
AND2, and codomain and AND are taken. As a result, the expanded
propagation set is propagated to Y. Here, the weak link from the map
node to the AND2 node is in an undetermined state due to the small
number of observations. The weak link is strengthened by positive
feedback. As a result, the probability of the expansion of the set in
map also approaches confirmation at the same time.

\subsubsection{General State Representation}\label{general-state-representation}

Using mapping, a large state can be expressed as shown below. Substates
X1, X2, etc. can be hierarchically connected to further states. It is
not only possible to divide and show each state by address, but also to
use continuous time series or coordinates, and it is also possible to
use destination nodes from other mappings or abstract nodes as
conditions.

\begin{figure}[ht]
  \centering
  \includegraphics[width=0.7\textwidth]{SOLE/DocumentImage/Fig_9.png}
  \caption{Multivariate state with mapping links}
  \label{fig:multivariate_state_with_mapping_links}
\end{figure}

\subsection{Stochastic variation distribution hierarchical feedback
algorithm}\label{stochastic-variation-distribution-hierarchical-feedback-algorithm}

To reproduce the observed probabilities by resolving the difference
between the propagation probability of a path through the network in SOL
and the probability of propagation of the path observed in the external
observation function. We consider this to be generalized learning in
SOL. The objective is almost identical to the learning of existing
ordinary neural networks, but without heuristics such as activation
functions, but in a more rigorous manner.

\subsubsection{Propagation, collisions, and
feedback}\label{propagation-collisions-and-feedback}

From any common starting node, it is possible to reach another identical
node again by passing through multiple paths in the network. For
example, a collision between a node with a result observed in the past
and a node with a result observed currently is a typical example.

If there is an overlap in the set of assumption vectors that are the
starting points of the two routes, it is considered a collision. If the
propagation probabilities of the two colliding routes are different,
feedback is performed. In order to accurately calculate the amount of
feedback, this probability fluctuation distribution hierarchical
feedback performs a probability calculation and identifies the links to
be fed back and the amount of feedback.

\begin{figure}[ht]
  \centering
  \includegraphics[width=0.7\textwidth]{SOLE/DocumentImage/SOL_ProbabilityPath.png}
  \caption{Difference in probability of passing two passes}
  \label{fig:difference_in_probability_of_passing_two_passes}
\end{figure}

\subsubsection{Calculating feedback
value}\label{calculating-feedback-value}

A difference in the propagation probabilities of the two paths occurs
when the propagation probabilities of the two paths as a whole are
observed. To resolve this difference, feedback is provided to each path.
The problem here is that with hierarchical links, it is difficult to
determine which and how much feedback to apply to which link. The
``stochastic variable distribution hierarchical feedback'' algorithm
solves this problem with a more precise method. This method makes it
possible to properly provide near-optimal probabilistic feedback for
each link in a theoretically unlimited hierarchy with a small
computational order of magnitude.

The basic concept starts from the fact that the probability of
occurrence of a probability fluctuation of link n depends on the
propagation probability \(P_n\) of link n and the number of observations
\(N_n\). As a result of passing through this link multiple times, the
total propagation probability \(P_{total}\) is observed. If a difference
occurs with the result of the total propagation probability, the
difference is distributed to each link according to the weight \(w_n\)
of each link and fed back as follows. The weight \(w_n\) of each link is
calculated by the following formula (it is based on a fairly strict
derivation, but it is omitted here). \(P_n\) is the propagation
probability of the link observed so far, and \(N_n\) is the number of
times feedback has been applied so far. \(E_n\) is the effect
coefficient for the effect of the probability fluctuation of each link
on the total propagation probability, and is calculated by the total
propagation.

\[  w_n=\frac{P_n(1-P_n)}{N_n} \]

\[P_n'=P_n+ w_n\frac{\Delta P_{total}}{ \sum_n E_n w_n} \]

The above \(W_n\) and \(E_n\) are used to calculate the probability of
the entire propagation. For \(E_n\), the probabilities of individual
links are applied and propagated according to the propagation of the
links.

\begin{figure}[ht]
  \centering
  \includegraphics[width=0.7\textwidth]{SOLE/DocumentImage/SOL_WeightPropagation.png}
  \caption{Propagation of weight value Wn and effect coefficient En}
  \label{fig:propagation_of_weight_value_wn_and_effect_coefficient_en}
\end{figure}

Once propagation is complete, the fluctuation in the overall observation
probability is distributed according to \(W_n\),\(E_n\) so that the
propagation probability of each link is as close to 0 or 1 as possible
and the sign of the fluctuation value is consistent.

The propagation probability corrected by feedback does not become the
next propagation probability of the link as it is. The larger the number
of feedback observations of the link up to that point, \(N_{n}\), the
smaller the amount of feedback of the link propagation probability.
\(N_{n}\) is added using the link's effectiveness coefficient, \(E_n\),
to become \(N_{n}'\).

\[ N_{n}'=N_{n}+E_n\]

\[P_n'=\frac{N_{n}P_n +E_nP_n'}{N_{n}'}\]

This stochastic variable distribution hierarchy feedback allows the
feedback to act more accurately on deep hierarchical networks, even when
compared to neural network backpropagation. The reason is the
application of rigorous probability theory.

\subsubsection{Forming associations from simultaneous observations}\label{forming-associations-from-simultaneous-observations}  

Simultaneous observation is the basic method for defining associations
between nodes that are collectively non-overlapping. When a subset of a
node has a value determined simultaneously with a subset of another
node, it is fundamental to associate them by mapping. Simultaneous means
that the assumption vectors are congruent or entailing. However, this
alone may result in a coincidental coincidence of unrelated events. For
this reason, feedback is used to increase the probability of
association.

Therefore, the mapping association is formed from the observed fact that
the probability P of occurrence of the two values is somewhat low and
that the two values are determined simultaneously. Typically, the
association is formed from the simultaneous variation of node X and node
Y.

\begin{figure}[ht]
  \centering
  \includegraphics[width=0.5\textwidth]{SOLE/DocumentImage/Fig_12.png}
  \caption{Forming associations from simultaneous observation}
  \label{fig:forming_associations_from_simultaneous_observation}
\end{figure}

The associative uncertainty is represented by the probabilities
\(P^{11}\) and \(P^{00}\) of the link from Map to AND1 or from Map to
AND2. \(N^{11}\),\(N^{00}\) of a link indicates the number of feedbacks
to that link.

Let \(P_X\),\(P_Y\) be the past observation probabilities of X and Y,
respectively. If \(Y=1\) when \(X=1\), the specific initial value of the
LinkY parameter is determined by the following formula. In particular,
the strength increases when \(P_Y\) is sufficiently small.

\[ P_y^{11}=1\]

\[ P_y^{00}=0.5\]

\[ N_y^{11}= -log_2 P_Y \]

\[ N_y^{00}= 0\]

If \(Y=0\) when \(X=0\), the parameters are determined by the following
formula.

\[ P_y^{11}=0.5\]

\[ P_y^{00}=1\]

\[ N_y^{11}= 0\]

\[ N_y^{00}= -log_2 (1-P_Y) \]

LinkX is defined similarly. Other links are considered to have definite
values. Links with fixed values are not subject to feedback.

After the association is formed, feedback is performed by observing the
propagation probability of both \(X=1\) to \(Y=1\) and \(X=0\) to
\(Y=0\). As a result, \(N_y^{11}\) and \(N_y^{00}\) of the associative
link are further added, and the probability of both \(P_y^{11}\) and
\(P_y^{00}\) sides is determined.

The objects X and Y that form the association are not only nodes of the
observation result, but also condition nodes that are the result of
feedback, link nodes that indicate causal relationships, etc., and it is
possible to form associations based on probability between any subset,
and it is possible to correspond to any abstract concept.

\subsection{Stochastic autonomous logic generation
algorithm}\label{stochastic-autonomous-logic-generation-algorithm}

SOL uses a ``probabilistic autonomous logic generation algorithm''. This
algorithm autonomously generates and modifies nodes and links.

\subsubsection{``Conditional'' Link splitting and logical
operation node
insertion}\label{conditional-link-splitting-and-logical-operation-node-insertion}

If the link probability approaches 1 to 0.5 or 0 to 0.5 as a result of
the feedback, the link is considered to have been negatively fed back.
In other words, the uncertain probability part of the link is separated
using logical operations with the current premise vector as the
condition.

By managing the link propagation probability as a binary pair of
\(P^{11}\),\(P^{00}\), the AND node or OR node to be inserted can be
deterministically selected based on the direction of the propagation
probability where the negative feedback occurred. If \(P^{11}\)
approaches 1 to 0.5, the node is an AND node, and if \(P^{00}\)
approaches 1 to 0.5, the node is an OR node. The XOR node is selected
when both \(P^{11}\) and \(P^{00}\) approach 0.5, but this is limited to
cases where feedback is given under the same conditions for both P11 and
P00. The condition link to be added to the inserted logical operation is
selected from nodes with the same premise vector.

\begin{figure}[ht]
  \centering
  \includegraphics[width=0.7\textwidth]{SOLE/DocumentImage/SOL_AddingCondition.png}
  \caption{Insert a condition AND into the link from A to B}
  \label{fig:insert_a_condition_cond_into_the_link_from_a_to_b}
\end{figure}

This is the basis for the autonomous generation of logic operations in
SOL, which enables the formation of accurate logic operations.

\subsubsection{``Causal'' Link node activation and associative
targeting from positive
feedback}\label{causal-link-node-activation-and-associative-targeting-from-positive-feedback}

If the probability of the activation value passing through a random link
matches the activation value of another fixed probability, the random
link is positively fed back, and as a result, the ``link node''
corresponding to that random link is considered to be activated by the
positive feedback.

The concrete entity of the link node is an unknown input node that is
input to a logical operation node that is virtually added and inserted
into the random link. An example is as follows. \(\land\) is the logical
operation AND.

\[B=A \land L\]

Back propagation to link nodes propagates the matching results of the
propagation probabilities \(P_{A}\) and \(P_{B}\) at both ends of the
link. Specifically, the propagation probability \(P_{L}\) is expressed
as follows:

\[ P_{L}=\frac{P_{B}}{P_{A}}\]

If the probability values \hspace{0pt}\hspace{0pt}before and after the
link are both the same, 1 and 0, the following link node is specially
generated. In this case, XOR\(\oplus\) is used.

\[B=A \oplus \neg L\]

The back propagation to the link node in the case of XOR is as follows.

\[ P_{L}=\frac{P_{A}+P_{B}-1}{2P_{A}-1}\]

This is the basic method for using the results of causal relations
between two different sets in logical operations. Even if the
probabilities of \(P_{A}\) and \(P_{B}\) are intermediate, if they match
at anything other than 0.5, the probability becomes 1.

In this way, when input A and output B match as premise sets, the link
node Match is activated by backpropagation. This link node is considered
to be a match between A and B.

\begin{figure}[ht]
  \centering
  \includegraphics[width=0.3\textwidth]{SOLE/DocumentImage/Fig_14.png}
  \caption{Representation of causal relationships by backpropagation}
  \label{fig:representation_of_causal_relationships_by_backpropagation}
\end{figure}

This can be thought of as a causal relationship node. Causal
relationship node can be applied to conditional flow control, for
example, by judging the match of numerical values. Backpropagation from
XOR indicates an exact match, while backpropagation from AND or OR
indicates an inclusion relationship.

\subsubsection{``Substitution'' Coupling between nodes that are
inclusive in terms of the propagation
set.}\label{substitution-coupling-between-nodes-that-are-inclusive-in-terms-of-the-propagation-set.}

When propagating using forward logical operations, the propagation set
becomes smaller as conditions such as AND and OR are added. Here, by
propagating another logical operation in the reverse direction, the
conditions such as AND and OR are removed, and the propagation set
before the conditions were applied can be restored and propagated. As a
result, a containment relationship may be established between two
distant mapping nodes. This propagation between consistent mapping nodes
is called assignment. In the following example, value A is included in
variable X (\(A \subset X\)), so Map1, which contains value A, becomes a
subset of Map2, which contains variable X, and Map1 is assigned to Map2.

\begin{figure}[ht]
  \centering
  \includegraphics[width=0.5\textwidth]{SOLE/DocumentImage/Fig_15.png}
  \caption{Assign between map nodes}
  \label{fig:assign_between_map_nodes}
\end{figure}

It is possible to perform logical operations using multiple conditions
within a state, and to use hierarchical states. It is similar to pattern
matching in Prolog, etc., and is a more generalized method.

\subsubsection{``Instantiation'' Deterministic partial network
replication}\label{instantiation-deterministic-partial-network-replication}

When an activation value passes through a network, it may collide with
another determined activation value, causing positive feedback and
determining part of the network. As a result, it becomes possible to
extract only the determined part. This is called instantiation.

In this case, if there are multiple branches in the link path, such as
output links or logical operation input links, the network of the path
is partially duplicated to create a kind of instance. The instance omits
unnecessary link branches.

An example is shown in the figure below. The State on the right side is
assigned to FunctionState as a subset in advance. From the assigned
State, it passes through the function and is propagated as the result of
function execution to the activation value indicating the value a. This
propagation set does not overlap with the State, but it is possible to
expand the propagation set of the State without contradiction. In this
way, the result of the function is duplicated and added to the input
state.

\begin{figure}[ht]
  \centering
  \includegraphics[width=0.7\textwidth]{SOLE/DocumentImage/SOL_NetworkInstantiate.png}
  \caption{Duplication of a network instance}
  \label{fig:duplication_of_a_network_instance}
\end{figure}

This ``instantiation'' is also used to assign function results.
Instantiation can extract parts of the network structure with further
determined probabilities, and at the same time, it has the effect of
reducing the number of propagation branchings, thereby reducing the
number of search branchings in propagation.

\subsubsection{``Generalization'' Generalized partial duplication
of a
network}\label{generalization-generalized-partial-duplication-of-a-network}

When an existing network is reused, a partially inconsistent part may be
fed back. For example, in the case of feedback to the P00 side of a
link, it is possible to partially expand and duplicate the network. This
is ``generalization.'' The biggest difference from instantiation is that
the expanded part is uncertain and needs to be confirmed in future
verification.

\begin{figure}[ht]
  \centering
  \includegraphics[width=0.7\textwidth]{SOLE/DocumentImage/SOL_NetworkGeneralization.png}
  \caption{Generalized duplication of a network}
  \label{fig:generalized_duplication_of_a_network}
\end{figure}

\subsubsection{``Selection'' Selective control of multiple link
propagation}\label{selection-selective-control-of-multiple-link-propagation}

There can be a large number of output links that can be connected to a
node. To make link search more efficient, it is necessary to select a
link from a large number of links for a condition. It is desirable to
have SOL determine and make this selection process more efficient.

Controlling link selection means generating a link selection node
corresponding to the selected link and making it the target of
association and activation propagation. As a result, it is possible to
associate the utility obtained from the activation value of the selected
link with the link selection node. Conversely, the link selection node
is activated from the utility, and the link corresponding to the utility
is selected.

Controlling the selection of the link selection node is different from
the method of controlling links by adding conditional logical
operations, and can maintain the link propagation probability. Even if a
link is not selected, the link propagation itself is not particularly
hindered, so propagation itself is possible, but the possibility of it
being selected as a propagation target is low.

\subsubsection{Stochastic autonomous logic
generation}\label{stochastic-autonomous-logic-generation}

These are the basic components of the probabilistic autonomous logic
generation algorithm. AND, OR, XOR, NOT, mapping links, and link control
can be automatically generated using the above methods. For probability
fluctuations and sets, definite probability links with a propagation
probability close to 100\% or 0\% are selected as much as possible. For
links with a large number of observations and uncertain propagation
probabilities, other conditions are added sequentially and replaced with
definite probability links as much as possible.

In other words, unlike neural networks that use analog value weights,
this probabilistic autonomous logic generation algorithm is an algorithm
that attempts to digitize and reproduce the observed object as much as
possible. Since definite digital logic is generated, it becomes possible
to output the learning results in the form of a logical formula.

\subsection{Autonomous Generation of Sequential
Circuits}\label{autonomous-generation-of-sequential-circuits}

\subsubsection{From association between fluctuations to
states}\label{from-association-between-fluctuations-to-states}

A generalized observation is an act of associatively connecting a time
node with multiple observation nodes observed at that time. For example,
multiple bit values \hspace{0pt}\hspace{0pt}and a time node are combined
to form a state.

\begin{figure}[ht]
  \centering
  \includegraphics[width=0.7\textwidth]{SOLE/DocumentImage/Fig_18.png}
  \caption{Form associations between time and state}
  \label{fig:form_associations_between_time_and_state}
\end{figure}

\subsubsection{Autonomous generation of logical operations between
states}\label{autonomous-generation-of-logical-operations-between-states}

V1, V2, V3, and V4 are Boolean values \hspace{0pt}\hspace{0pt}that are
observations at a certain time. Each observation is different depending
on the time.

\begin{figure}[ht]
  \centering
  \includegraphics[width=0.7\textwidth]{SOLE/DocumentImage/Fig_19.png}
  \caption{Associate with time passage and state}
  \label{fig:associate_with_time_passage_and_state}
\end{figure}

For the network of V1 to V4 connected by states, propagation occurs
again from the second observation from V1 to V4. Feedback occurs due to
a collision of probabilities between State and V1, generating StateAND1,
and State is split into State1 and State2, which become the input
condition for StateAND1. As a result, the following logical formula is
formed.

\[ V_3=V_1 \land V_2\]

\subsubsection{Forming associations in chronological
order}\label{forming-associations-in-chronological-order}

\begin{figure}[ht]
  \centering
  \includegraphics[width=0.4\textwidth]{SOLE/DocumentImage/Fig_20.png}
  \caption{Associations between time variability and state}
  \label{fig:associations_between_time_variability_and_state}
\end{figure}

Before and after the Time1 node, StateX1 and StateX2 are combined in an
exclusive and continuous time series by the variation association. This
results in an association between StateX1 and StateX2 at adjacent times.
Furthermore, the link from States to AND2 is weak and requires a
condition, which forms a logical expression between States to form an
sequencial circuit.

\subsubsection{Generic Function
Generation}\label{generic-function-generation}

Using the above elements, a sequential circuit is generated
autonomously. Furthermore, by generalizing and duplicating the learned
state, a generic function is generated in which the values
\hspace{0pt}\hspace{0pt}before and after the mapping are generalized
into variables.

The value nodes that are the subject of the mapping are Boolean values
\hspace{0pt}\hspace{0pt}in the examples so far, but any object can be
linked, such as more abstract numbers, coordinates, and character
tokens. Interactions such as calculations between abstract nodes are
realized using built-in functions defined outside SOL, and SOL selects
the links between the built-in functions during learning.

\section{Bidirectional logic operations}\label{bidirectional-logic-operations}

\begin{enumerate}
\def\labelenumi{\arabic{enumi}.}

\item
  There are multiple real sets that divide space-time, and the
  containment relationship between the sets is unknown.
\end{enumerate}

A real set corresponds to the material of each space-time, and has a
certain extent of extension in the space-time direction. Multiple real
sets can be exclusive, contain, or overlap. Relationships between all of
these real sets can be derived by mapping.

\begin{enumerate}
\def\labelenumi{\arabic{enumi}.}
\setcounter{enumi}{1}

\item
  Among multiple real sets, things that are observed simultaneously in a
  specific situation can be connected as a ``map.''
\end{enumerate}

There are cases where a part of a real set is always observed
``simultaneously'' with a part of another real set. ``Simultaneous'' in
this case means selecting a subset that is unrelated to the real set. In
this case, a containment set that includes both subsets can be formed as
a ``map,'' regardless of the overlap relationship of the real sets
themselves. This containment set is considered a ``map set.'' It can be
expected that this ``map set'' expresses what is generally called a
causal relationship. The source of the map is completely included in the
map set, but the destination of the map may only include a subset of the
map set. This is due to the definition of ``implication.''

\begin{enumerate}
\def\labelenumi{\arabic{enumi}.}
\setcounter{enumi}{2}

\item
  Performing logical operations between multiple mappings.
\end{enumerate}

By performing a set logical operation between multiple mapping
destinations, the set resulting from the logical operation becomes a
subset of the final mapping destination. This is a logical operation
generalized by mapping.

It is assumed that the real space-time has this kind of structure. Based
on this, the aim is to reproduce the structure of real sets and mappings
using SOL from observation, whatever that may be.

\subsection{Nodes and Links}\label{nodes-and-links}

SOL is a bidirectional network consisting of nodes and links. All of the
following types of nodes are connected by links.

\begin{enumerate}
\def\labelenumi{\arabic{enumi}.}

\item
  Value nodes, which are the entities of observed values
\item
  Logic nodes, which indicate logical operations between links
\item
  Joint nodes, which connect links equivalently
\item
  Exclusive nodes, which indicate exclusivity between links
\item
  Function nodes, which connect to links and input and output to the
  outside
\end{enumerate}

Value nodes are abstract subsets that exist in space, but it is also
possible to have a one-to-one correspondence between nodes and actual
values \hspace{0pt}\hspace{0pt}such as scalars, vector values, and
characters.

Logical operation nodes perform Boolean algebraic operations on input
links. There are types such as AND, OR, and XOR. Logical operations such
as AND, OR, and XOR apply both logical operations and set operations to
multiple inputs. Output links not only propagate the results of logical
operations, but also indicate the equivalence of values
\hspace{0pt}\hspace{0pt}and sets between output links.

Connection nodes are nodes that only have outputs, and indicate the
equivalence of values \hspace{0pt}\hspace{0pt}and sets between multiple
links. Equivalence of multiple links also means that feedback occurs
when arriving from multiple links.

Exclusive nodes are nodes that indicate exclusivity between input links.
They are roughly equivalent to NOT links between all input nodes, but
are used for efficiency.

Function nodes use the actual values \hspace{0pt}\hspace{0pt}indicated
by value nodes to perform actual operations and external input/output.
They generate value nodes such as numbers that correspond to the results
each time.

A link consists of a binomial set of probabilities
\(P^{f11}\),\(P^{f00}\) to apply when traversing the link, and
experience numbers \(N^{f11}\),\(N^{f00}\). When drawing links, they
should only have arrows if the link destination is an input to a logical
operation. Otherwise, they are considered equivalent links and no arrows
are used.

\begin{figure}[ht]
  \centering
  \includegraphics[width=0.7\textwidth]{SOLE/DocumentImage/SOL_Links.png}
  \caption{Link Types}
  \label{fig:link_types}
\end{figure}

The attributes of the link will be as follows.

\begin{longtable}{@{}p{0.6\textwidth}p{0.3\textwidth}@{}}
  \caption{Probabilities related to link propagation} \\
  \toprule
  Content & Symbol \\
  \midrule
  \endfirsthead
  
  \toprule
  Content & Symbol \\
  \midrule
  \endhead
  
  \bottomrule
  \endfoot
  
  Forward propagation probability of the link & \(P^{f11}\), \(P^{f00}\) \\
  Forward experience probability of the link & \(N^{f11}\), \(N^{f00}\) \\
  Backward propagation probability of the link & \(P^{r11}\), \(P^{r00}\) \\
  Backward experience probability of the link & \(N^{r11}\), \(N^{r00}\) \\
\end{longtable}

Activation calculates the propagation probability from
the origin through multiple links and logical operations. 

\subsection{Link Propagation and Logical Operations}\label{link-propagation-and-logical-operations}

\subsubsection{Propagation}\label{propagation}

Link propagation is calculated using the propagation probabilities
\(P^{11},P^{00}\).

\[ P' = P^{11} P+(1-P^{00})( 1-P )\]

\subsubsection{NOT Propagation}\label{not-propagation}

If the link propagation probability is set as follows, the link itself
will behave as a logical NOT.

\[P^{11}=0\]

\[P^{00}=0\]

\subsubsection{AND Operation}\label{and-operation}

The results of link propagation are combined with an AND logical
operation. \(P_1\),\(P_2\)\ldots{} are input, and \(P'\) is output.

\[ P'= {P}_{1} {P}_{2} {P}_{3} ...{P}_{n} \]

\subsubsection{OR Operation}\label{or-operation}

The results of link propagation are combined with an OR logical
operation. \(P_1\),\(P_2\)\ldots{} are input and \(P'\) is output.

\[ P' = 1 - (1-{P}_{1})(1-{P}_{2})(1-{P}_{3})... (1-{P}_{n}) \]

\subsubsection{XOR operation}\label{xor-operation}

The results of link propagation are combined with XOR logical operation.
\(P_1\),\(P_2\)\ldots{} are input and \(P'\) is output.

\[ P' = P_1(1-P_2) + (1-P_1) P_2\]

For XOR of three or more variables, the operation is applied
recursively.

\subsubsection{Exclusive operation}\label{exclusive-operation}

The results of link propagation are input to the Exclusive operation. In
this case, propagation is performed without applying the operation.

\[ P'=P_n \]

\subsubsection{Reverse Propagation}\label{reverse-propagation}

Reverse propagation probabilities \(P_{r}^{11},P_{r}^{00}\) are defined
for each link.

For normal links, they can be calculated immediately from the forward
propagation probability. The forward propagation probabilities are
\(P^{f11}\) and \(P^{f00}\), and the reverse propagation probabilities
are \(P^{r11}\) and \(P^{r00}\). In this case, the reverse propagation
probabilities \(P^{r11}\) and \(P^{r00}\) are calculated using the
following formula.

\[ P^{f11}P^{r11}+(1-P^{f11})(1-P^{r11})=1\]

\[ P^{r11}=\frac{P^{f00}}{P^{f11}+P^{f00}-1} \]

\[ P^{r00}=\frac{P^{f11}}{P^{f11}+P^{f00}-1} \]

In the case of backpropagation from logical operations such as OR and
AND, it is calculated according to Bayes' theorem. In addition to the
normal link information, the link's source probability is \(P\) and the
destination probability is \(P^{'}\). If both the source and destination
probabilities are not available, the backpropagation probability cannot
be calculated.

\[ P^{r11}=\frac{P P^{f11}}{P'}\]

\[ P^{r00}=\frac{(1-P) P^{f00}}{1-P'}\]

\subsubsection{Entropy and link determination}\label{entropy-and-link-determination}

Entropy for probability can be calculated from the binary entropy
formula.

\[ S =\sum_n P_nlogP_n \]

When this is binarized,

\[ S = P log P + (1-P) log (1-P) \]

As a result, the probability P close to 1 or 0 has the smallest entropy
S, and the probability P close to 0.5 has the largest entropy S. The
objective of SOL is to make the link probability as close to 1 or 0 as
possible. In other words, minimizing entropy is one of the objectives of
SOL's self-organization. To achieve this, links with high entropy are
split by inserting appropriate logical operations to reduce the entropy
of each link.

\subsubsection{Feedback to links}\label{feedback-to-links}

Feedback to the link separates the links vertically into multiple links
for each set that passes through the link. Positive feedback, where the
overall probability approaches 1 or 0, does not require separation.
Negative feedback, where the overall probability approaches 0.5, is
considered to be a mixture of subsets with propagation probabilities of
1 and 0 among the sets that pass through the link.

For example, if a link previously passed with probability 1, but now
passes with probability 0, the propagation set with probability 0 that
passed is considered to be a subset that separates the link vertically.
The means of this vertical division is the insertion of logical
operations.

\subsection{Vertical splitting of links and insertion of logical
operations}\label{vertical-splitting-of-links-and-insertion-of-logical-operations}

Links propagating from node to node can be split vertically by splitting
the starting node and the ending node into subsets.

A link is split when it is observed that there is a set with a different
propagation probability among the sets passing through the link. When
the propagated observation probability approaches 0.5, the link is
separated into an element with an observation probability of 1 and an
element with an observation probability of 0. The elements of the
separated link use the information of the starting node that indicates
the set of elements to form a logical operation with that node.

\begin{itemize}

\item
  If the \(P^{11}\) side is uncertain and the \(P^{00}\) side is
  confirmed as 1, insert AND.
\item
  If the \(P^{00}\) side is uncertain and the \(P^{11}\) side is
  confirmed as 1, insert OR.
\item
  If the \(P^{11}\) side is uncertain and the \(P^{00}\) side is
  confirmed as 0, insert NOT(AND).
\item
  If the \(P^{00}\) side is uncertain and the \(P^{11}\) side is
  confirmed as 0, insert NOT(OR).
\item
  If both the \(P^{11}\) side and the \(P^{00}\) side are uncertain,
  there is a possibility of inserting XOR.
\end{itemize}

\section{Mapping}\label{mapping}

\subsection{Definition of Mapping}\label{definition-of-mapping}

Logical operations can be described as the relationship between sets.
However, to describe more general logic, memory elements such as
flip-flops and memory that can switch between various time states are
essential. Otherwise, it is impossible to create general-purpose logical
operations for fluctuations in time and space.

SOL adds the concept of mapping to logical operations. Mapping connects
different states in time and space and uses them as new starting points.
The action of this mapping is itself a generalization of memory
elements. It also makes it possible to manage the agreement of multiple
states as a state. This makes it possible to use the relationship
itself, such as the order between states, as a state.

This shows how to express mapping using only logical operations. In the
following example, a subset X of the source set is mapped to a subset Y
of the destination set. A Map node is a type of connection node.

\begin{figure}[ht]
  \centering
  \includegraphics[width=0.5\textwidth]{SOLE/DocumentImage/Fig_22.png}
  \caption{Map with weak links}
  \label{fig:map_with_weak_links}
\end{figure}

A weak link is a link with a small number of observations, either
\(N^{11}\) or \(N^{00}\), and can become uncertain due to negative
feedback from the next observation. Further logical operations are added
to make this link certain. Logical operations can be added to either of
the two weak links, but only the logical operation that is consistent
will be certain.

\subsection{Back propagation and substitution through the map}\label{back-propagation-and-substitution-through-the-map}

\begin{figure}[ht]
  \centering
  \includegraphics[width=0.7\textwidth]{SOLE/DocumentImage/SOL_MapSubstitution.png}
  \caption{Substitution between maps}
  \label{fig:substitution_between_maps}
\end{figure}

This diagram shows a method for sharing the source and destination of
multiple mappings. From the perspective of the Map1 node, X is the range
of the mapping, and from the perspective of the Map2 node, X is the
domain of the mapping. The important thing is that the Cond node that is
the condition for the range of the Map1 node is the same as the Cond
node that is the condition for the domain of the Map2 node.

First, the propagation from the map1 node to the map2 node is a
backpropagation. The activation value at the map1 node is 1. However, by
using the concept of activation value substitution, we will show why the
probability of the propagation set reaching the Map2 node is uniformly
1.

The activation value with probability 1 that exists on the Map1 node is
propagated from the Map1 node to X. At that time, AND is taken at the
node Cond, and the propagation set is divided. However, when it is
propagated from X to the Map2 node, it passes through the
backpropagation of AND. At that time, it is complemented by the same
Cond node as the Map1 node. As a set, it becomes the same size as the
activation value of the Map1 node. The activation value of the Map2 node
cannot be completely determined to be either 0 or 1. However, since it
is the backpropagation of AND from the value 1, it is possible to assume
that all values \hspace{0pt}\hspace{0pt}of the activation value of the
Map2 node are 1 in the propagation set.

As a result, we can assume that the activation value in the Map1 node is
propagation-set-wise identical to the activation value in the Map2 node.
This is considered to be an assignment in SOL. The conditions for this
are as follows:

\begin{enumerate}
\def\labelenumi{\arabic{enumi}.}
\item
  There is no contradiction because the value of the propagation source
  and the value of the determined part of the propagation destination
  are the same.
\item
  The assumption vectors of the two propagation sets are the same.
\end{enumerate}

This assignment is similar to Occam's razor and is strictly speaking
unfounded. However, the validity of this assignment is verified between
the propagation using the assigned value and the observation.
Furthermore, the conditions for validity are added by feedback.

\subsection{Hierarchical matching by mapping and bidirectional
propagation}\label{hierarchical-matching-by-mapping-and-bidirectional-propagation}

As an example, let us take text analysis from Input. In the following
example, three mutually exclusive words are activated using mutually
exclusive nodes L1, L2, and L3. To determine whether they match,
backpropagation from the words occurs, and InputAND is activated in
reverse. It is sufficient to confirm that L1, L2, and L3 are mutually
exclusive. Backpropagation from AND1 to Map1 then expands them
collectively and they are no longer exclusive. The ANDs from Map1, 2,
and 3 are combined to activate InputAND. Conversely, InputAND will not
be activated if all elements are missing.

In this way, using bidirectional propagation makes it possible to
perform set logic operations between words that collectively do not
overlap.

\begin{figure}[ht]
  \centering
  \includegraphics[width=0.7\textwidth]{SOLE/DocumentImage/Fig_24.png}
  \caption{Natural language expression using mapping}
  \label{fig:natural_language_expression_using_mapping}
\end{figure}

Furthermore, the concepts of time series and distance are added to these
exclusive nodes L1, L2, and L3 to represent actual text sentences.

\subsection{Order and Natural Language}\label{order-and-natural-language}

\subsubsection{Representation method using mapping nodes and order
links}\label{representation-method-using-mapping-nodes-and-order-links}

This shows a method for expressing order using sets and mappings.
PreviousState and CurrentState are states of successive time, and are
linked by a higher-level mapping StateMap. PreviousState and
CurrentState are linked by the conditions Previous1 and Next1, which
indicate the order, respectively.

\begin{figure}[ht]
  \centering
  \includegraphics[width=0.5\textwidth]{SOLE/DocumentImage/Fig_25.png}
  \caption{Representing the order between states by mapping}
  \label{fig:representing_the_order_between_states_by_mapping}
\end{figure}

Previous1 and Next1 are the condition nodes used for this StateMap1, but
we also consider this to be a subset of the generalized ordering
Previous, Next nodes. This grouping can be used for substitution into
the ordering of another network.

\begin{figure}[ht]
  \centering
  \includegraphics[width=0.7\textwidth]{SOLE/DocumentImage/Fig_26.png}
  \caption{Grouping by rank}
  \label{fig:grouping_by_rank}
\end{figure}

\subsubsection{Sentence Matching}\label{sentence-matching}

In the following example, three mutually exclusive words are activated
using the mutually ordered links Next1 and Next2. Next1 and Next2 are
also link nodes that are activated when the order between the words is
established. Matching is input from the link nodes Next1 and Next2 to
InputAND. Unlike the previous example, if the order of the words is
swapped, the link nodes Next and Next2 will not be activated. In that
case, no substitution is made to the map that represents the entire
sentence.

\begin{figure}[ht]
  \centering
  \includegraphics[width=0.7\textwidth]{SOLE/DocumentImage/Fig_27.png}
  \caption{Representing a sentence as a mapping order}
  \label{fig:representing_a_sentence_as_a_mapping_order}
\end{figure}

This ordered sentence network is reused. The next time a partially
identical sentence is input, it is propagated to this existing network.
As a result, the state of the long sentence is divided and conditions
are added along the way. In this way, the sentence state is
hierarchical, logical operations are added, and it is associated with
other sentences and other concepts.

This method of expressing the order may seem inefficient, but it is a
strict method that is not affected by the order structure in space and
time. Many current machine learning methods implement the order as a
simple vector, but one-dimensional vectors have limited versatility.

\section{Bidirectional propagation of activation values}\label{bidirectional-propagation-of-activation-values}

SOL propagates activation values \hspace{0pt}\hspace{0pt}in a similar
way to neural networks, but strictly manages the propagation probability
from the starting point. In addition, it manages the information of the
sets propagated by the assumption vector. In addition, SOL allows the
propagation of logical operations in both directions to express
mappings. This makes it possible to calculate the propagation
probability and propagation set between any sets in space as long as
they are connected by a network.

For the links in the SOL network, the following Activation objects,
which indicate activation values, are distributed and propagated for
each link, generating activation hierarchically. The information held by
activation is basically as follows.

\begin{enumerate}
\def\labelenumi{\arabic{enumi}.}

\item
  Propagated probability
\end{enumerate}

The propagated probability value is a scalar value between 0 and 1.

\begin{enumerate}
\def\labelenumi{\arabic{enumi}.}
\setcounter{enumi}{1}

\item
  Assumption vector
\end{enumerate}

An assumption vector that indicates multiple propagation sets that are
the starting points of this activation.

\subsection{Propagation of activation values}\label{propagation-of-activation-values}

\begin{enumerate}
\def\labelenumi{\arabic{enumi}.}

\item
  Probability propagation from assumptions
\end{enumerate}

The activation value starts from a hypothesis. The hypothesis is that
the probability of the starting node is either 1 or 0. The activation
value is propagated by applying link propagation and logical operations
to this starting probability.

\begin{enumerate}
\def\labelenumi{\arabic{enumi}.}
\setcounter{enumi}{1}

\item
  Logical operations between assumption vectors and integration of
  multiple assumption vectors
\end{enumerate}

When performing logical operations using multiple inputs with different
assumption vectors, only the overlapping parts of the sets indicated by
the assumption vectors are extracted and propagated. Logical operations
are not applied to parts where the assumption vectors do not overlap.

\begin{enumerate}
\def\labelenumi{\arabic{enumi}.}
\setcounter{enumi}{2}

\item
  Backpropagation of logical operations and complementary synthesis of
  assumption vectors
\end{enumerate}

\begin{itemize}

\item
  The set on the output side of the logical operation is propagated in
  reverse to the input side of the logical operation.
\item
  When backpropagating a value of 1 from AND, the probability 1 and the
  assumption vector are propagated to all inputs as is.
\item
  When backpropagating a value of 0 from AND, if there is an input where
  all inputs are 1 except for one, the input is determined to be 0.
\item
  When a value of 0 is backpropagated from AND, if there is already a 0
  in the input, the probability of the backpropagated input is
  uncertain. However, there is another way to make the uncertain
  probability certain. This is why backpropagation can be used for
  mapping.
\end{itemize}

\begin{enumerate}
\def\labelenumi{\arabic{enumi}.}
\setcounter{enumi}{3}

\item
  Activation collision and comparison of elements of assumption vectors
\end{enumerate}

When multiple activation values \hspace{0pt}\hspace{0pt}reach the same
node via different paths, if there is a common part between the two
assumption vectors, the probabilities of both are compared. If the
probabilities are different, feedback is performed to make the
probabilities equal.

\begin{enumerate}
\def\labelenumi{\arabic{enumi}.}
\setcounter{enumi}{4}

\item
  Built-in Functions and Observations
\end{enumerate}

When the activation value reaches the built-in function, external
observations and external actions are performed. Observation means
connecting the input assumption vector and the output node with a
mapping.

\subsection{Assumption Vector}\label{assumption-vector}

\subsubsection{Assumption Vector Elements}\label{assumption-vector-elements}

Each element of the assumption vector represents an assumption that the
propagation probability of the link from a particular node to a node,
\(P^{11}\) or \(P^{00}\), is either 1 or 0. By assuming that the node
value is 1 or 0, the propagation set beyond that point is divided into
two.

\begin{enumerate}
\def\labelenumi{\arabic{enumi}.}

\item
  Source link
\end{enumerate}

The elements of the assumption vector assume that the propagation
probability of the source link is either 0 or 1.

\begin{enumerate}
\def\labelenumi{\arabic{enumi}.}
\setcounter{enumi}{1}

\item
  Backpropagation selection
\end{enumerate}

Backpropagating the input link from the activation value of 1 to the OR
node, one of its input nodes is selected. This selection action can be
considered as an assumption of the link from the OR node to the input
node. The same is true for backpropagation from the activation value of
0 to the AND node.

\begin{enumerate}
\def\labelenumi{\arabic{enumi}.}
\setcounter{enumi}{2}

\item
  Common origin in the middle of the network
\end{enumerate}

When multiple propagations collide, it is wasteful to consider all the
assumption vectors of the route. Therefore, an assumption is set for the
common origin of multiple propagations, and the assumptions of the
common paths before that are not compared.

\subsubsection{Interaction of logical
operations}\label{interaction-of-logical-operations}

When logical operations such as AND and OR are performed between
propagation sets, the assumption vectors are combined. The following
rules apply to this combination.

\begin{itemize}

\item
  Assumption vector elements that are different between assumption
  vectors are added to the destination as is.
\item
  Identical assumption vector elements between assumption vectors are
  aggregated into one.
\item
  Composition with assumption vector elements that are collectively
  inclusive replaces the smaller element in the propagation set.
\item
  Composition with assumption vector elements that are collectively
  exclusive (different starting point probabilities) causes the
  propagation set itself to become an empty set.
\end{itemize}

\subsubsection{Activation value collisions and assumption
vectors}\label{activation-value-collisions-and-assumption-vectors}

If multiple activation values \hspace{0pt}\hspace{0pt}reach the same
connection node or logical operation node, they are considered
equivalent and will collide. In terms of logical operations, the result
of combining the inputs and the output will collide.

If there is an overlap in the propagation set indicated by the
assumption vector, the probability values \hspace{0pt}\hspace{0pt}of the
propagated results must match. If they do not match, feedback is
executed.

\subsection{Probability Propagation of Activation Values}\label{probability-propagation-of-activation-values}

The basis of SOL is to propagate activation values
\hspace{0pt}\hspace{0pt}along links, propagate probabilities according
to the link propagation probability and node probability combination,
and execute the built-in function at the end of the link.

\(P^{f11}\) is the propagation probability from P with probability 1 to
P' with probability 1, and \(P^{f00}\) is the propagation probability
from P with probability 0 to P' with probability 0. This is a trivial
application of probability theory.

\[ P'=P P^{f11} +(1-P)(1-P^{f00})\].

In the AND operation, the probabilities of multiple activations are
integrated and propagated to the output link. The probability
calculation is performed using multiple input propagation probabilities
\(P_{1}\) , \(P_{2}\) , \(P_{3}...\) as follows:

\[ P'=P_{1} P_{2} P_{3}... \]

The probability calculation for the OR operation is as follows:

\[ P'=1-\{ (1-P_{1}) (1-P_{2}) (1-P_{3}) ... \} \]

The probability calculation for the XOR operation is as follows. In the
case of three or more inputs, the following formula is applied
recursively:

\[ P'= P_{1} (1-P_{2})+ (1-P_{1})P_{2} \} \]

When the number of output links is huge, activation is propagated only
to a few links. The method for limiting the links will be provided
separately.

\subsection{Backward propagation of probability}\label{backward-propagation-of-probability}

Activation can also be performed by backpropagation, which follows the
opposite direction of the links. This has a different meaning from
backpropagation in neural networks.

Backpropagation is performed in the same way as the forward propagation
probability calculation, using the backpropagation probabilities
\(P^{r11}\),\(P^{r00}\) for each link. No logical operations are
applied.

\[ P'=P P^{r11} +(1-P)(1-P^{r00}) \]

As a result, the total propagation probability is generalized as a
polynomial for multiple links as follows:

\[ P=f(P_1^{f11},P_1^{f00},P_2^{r11},P_2^{r00},P_3^{f11},... )\]

\subsection{Collision of active values}\label{collision-of-active-values}

A collision of activation values \hspace{0pt}\hspace{0pt}refers to the
difference in the propagation probability even though the propagated
hypothesis vectors overlap on multiple propagation paths that start from
the same node and reach the same node.

In principle, \(P=0.5\), that is, partial overlap with a propagation set
with an uncertain probability, is not a collision because it can be
considered unrelated as a propagation set. Conversely, if both
probabilities are certain, overlap in the propagation set can be
considered a probability collision. If the side with probability 1 (or
0) completely contains probability 0.5 in terms of the propagation set,
a contradiction occurs in the certainty of the side with probability 1,
so it is considered a collision.

The sizes of the propagation sets are compared using the above methods,
and probability collision and feedback are applied to the parts where
the propagation sets completely overlap.

\section{Hierarchical Feedback}
\label{hierarchical-feedback}

This feedback method is a newly developed method that can accurately
discover which links are causing errors in the observation probability
for an unlimited number of layers in a Bayesian network.

In conventional neural networks, it is possible to affect deep link
layers using backpropagation, but it is still difficult to identify the
cause of errors in deeper link layers. What is called deep learning is
simply a relatively deep hierarchy.

The assumptions behind this feedback are as follows:

\begin{enumerate}
\def\labelenumi{\arabic{enumi}.}

\item
  Each link in SOL has a propagation probability, and logical operations
  are also probability calculations. The overall propagation probability
  is calculated after passing through multiple links and logical
  operations. The observed new propagation probability \(P'\) is fed
  back to \(\Delta P_n\) for each link to bring the network propagation
  probability closer.
\end{enumerate}

\[P'=P_1 P_2 ... (P_n+\Delta P_n) ... P_{x-1}P_{x}=P_n+E_n\Delta P_n ...\]

\begin{enumerate}
\def\labelenumi{\arabic{enumi}.}
\setcounter{enumi}{1}
\item
  The fluctuation occurrence probability for each link with probability
  \(P\) to become probability \(P'\) depends on the past observation
  probability \(P\) and the number of observations \(N\), and can be
  calculated using pure probability theory.
\item
  The fluctuation occurrence probability for the entire link is the
  product of the fluctuation occurrence probabilities for all links.
  This overall fluctuation occurrence probability is maximized.
\item
  The number of observations \(N\) is added to each link for each
  feedback. The existing propagation probability is corrected according
  to the number of observations using the fed back propagation
  probability. Links with a large number of observations will have
  little correction due to feedback.
\end{enumerate}

A calculation method was determined to satisfy the above conditions.
This calculation method is called probability fluctuation distribution
hierarchical feedback.

\subsection{Binomial Propagation on
Links}\label{binomial-propagation-on-links}

The propagated activation value has one propagation probability \(P\).
This probability \(P\) is the probability of the current set of
hypothesis vectors. The probability of being in the complement is
\((1-P)\).

When an activation value passes through a probabilistically determined
link, the probability is transmitted according to the following
propagation probability.

\[ P^{11}=1 \]

\[ P^{00}=1 \]

NOT links act on the probability of the hypothesis vectors. Note that
they do not invert the set of hypothesis vectors that pass through.

\[ P^{11}=0 \]

\[ P^{00}=0 \]

A link has parameters \(N^{11}\) and \(N^{00}\) that indicate the
strength of the link, in addition to \(P^{11}\) and \(P^{00}\). This
strength is added for each feedback and determines the probability of
the link.

\subsection{Bidirectional propagation and probability calculation of
logical
operations}\label{bidirectional-propagation-and-probability-calculation-of-logical-operations}

Here is the method of calculating the probability in the activation
value propagation in SOL. The normal calculation method for the
propagation probability of a link is as follows.

\[ P' = P P^{11} + (1-P)(1-P^{00}) \]

When performing an AND operation, it is as follows. This is the
propagation probability where P=1.

\[ P_{1}'= P_{1}^{1} P_{1}^{2} P_{1}^{3} ...\]

OR operation

\[ P_{1}'= 1-P_{0}^{1} P_{0}^{2} P_{0}^{3} ...\]

The backpropagation probability of a link is calculated by applying
Bayes' theorem from the forward probability of the link.
\(P_{1}\),\(P_{0}\) are the probabilities of node A.
\(P^{f11}\),\(P^{f00}\) are the forward propagation probabilities from
node A to B. \(P^{r11}\),\(P_{r00}\) are the backward propagation
probabilities from node B to node A.

\[ \{ P^{f11}+(1-P)(1-P^{f00}) \} P^{r11}=P_{1}P^{f11}\]

\[ P^{r11}=\frac {P^{f11}}{ P^{f11}+P_{0}(1-P^{f00}) }\]

\[ \{ P^{f11}(1-P^{f11})+(1-P)P^{f00} \} P^{r00}=(1-P)P^{f00}\]

\[ P^{r00}=\frac {1-P^{f00}}{P^{f11}(1-Pf^{11})+(1-P)P^{f00}}\]

Backward propagation from logical operations such as AND and OR
operations to the input is also calculated using Bayes' theorem.
However, it is necessary to use the observation probability P(A) of the
link input and the observation probability P(B) of the logical operation
output.

\[ P^{r11}=\frac{P(A) P^{f11}}{P(B)}\]

\[ P^{r00}=\frac{(1-P(A)) P^{f00}}{1-P(B)}\]

As described above, the propagation probability synthesis for each node
in SOL is faithful to probability theory and is self-evident. Nonlinear
elements such as sigmoid functions are not used.

\subsection{What does feedback apply
to?}\label{what-does-feedback-apply-to}

The feedback is to adjust \(P_{total}\) calculated from the propagation
probability of the entire network currently in use to the propagation
probability \(P_{total}'\) of the colliding partner, which has a
different probability in the same situation.

SOL feedback is applied to each of the links \(P^{11}\) and \(P^{00}\)
through which the two colliding paths pass. The strength of the feedback
varies depending on the link probability and the number of links
experienced.

The feedback results \(P^{11'}\) and \(P^{10'}\) are corrected using
correction values \hspace{0pt}\hspace{0pt}\(\Delta P^{11}_n\) and
\(\Delta P^{00}_n\). The total probability of the two paths is
calculated by propagating it as a polynomial with a correction value
added to each. This results in a large-scale equation.

\[ P^{11'}_n=P^{11}_n+\Delta P^{11}_n \]

\[ P^{00'}_n=P^{00}_n+\Delta P^{00}_n \]

\[ P_{total}=(P^{11'}_1)(P^{11'}_2)(1-P^{11'}_3)... \]

The total probability of each of the two paths A and B is a propagation
polynomial for a large number of \(\Delta P^{11}_n\) and
\(\Delta P^{00}_n\). Furthermore, since the equations for these two
paths are equal,

\[ P_{totalA}-P_{totalB}=\Delta P_{total}=0 \]

All we need to do is find N \(\Delta P_{n}\)s that satisfy this equation
and have the smallest probability of variation. However, a proper method
would involve varying each \(\Delta P_{n}\) to find a solution, which
would result in a combinatorial explosion problem.

To avoid this, we provide a means to approximately find the amount of
variation \(\Delta P_{n}\) in each probability observation probability.

\subsection{Calculation of effect coefficient}\label{calculation-of-effect-coefficient-e_n}

The following is an equation that shows the extent to which the link
propagation probability fluctuation \(\Delta P_n\) affects the final
probability \(P_n\). The probability that affects the probability
fluctuation \(\Delta P_n\) is \(E_n\), and \(R_n\) is a constant term.

\[ P_n = R_n+ E_n\Delta P_n \]

This probability \(P_n\) changes when further links or logical
operations are applied. For this purpose, we will show an equation that
inputs \(E_n\) and \(R_n\) and outputs \(E'_n\) and \(R'_n\). The link
propagation probabilities are \(P^{11}\) and \(P^{00}\), respectively.

When the \(P^{11}\) side of a link is used as the starting point, the
equation is as follows. The propagation probability to the link is
\(P_{n-1}\).

\[ R_n+E_n\Delta P^{11}=(1-P^{00})(1-P_{n-1})+P_{n-1} \Delta P^{11}\]

\[ R_n = (1-P^{00})(1-P_{n-1})\]

\[ E_n=P_{n-1} \]

When the \(P^{00}\) side of the link is used as the starting point, the
formula is as follows.

\[ R_n+E_n\Delta P^{00}=P^{11}P_{n-1}+1-P_{n-1}+(P_{n-1}-1)\Delta P^{00}\]

\[ R_n = P^{11}P_{n-1}+1-P_{n-1}\]

\[ E_n=P_{n-1}-1\]

Propagation through links that are not the origin is expressed by the
following formula.

\[ R'_n+E'_n\Delta P_n=1-P^{00}+(P^{11}+P^{00}-1)Rn + (P^{11}+P^{00}-1)( E_n\Delta P_n)\]

\[ R'_n=1-P^{00}+(P^{11}+P^{00}-1)Rn \]

\[ E'_n= (P^{11}+P^{00}-1)E_n\]

Next, we calculate the propagation probability for the logical
operation. From now on, \(P^{11}\) and \(P^{00}\) are regarded as the
m-th and n-th links, respectively, and are denoted as \(P_{m}\) and
\(P_{n}\). The composite probability of inputs other than \(\Delta P_n\)
is \(P\). \(E'_n\) in AND operation is

\[ R'_n + E'_n\Delta P_n=PR_n+PE_n \Delta P_n\]

\[ R'_n = PR_n\]

\[ E'_n=PE_n\]

\(E'_n\) in OR operation is

\[ R'_n + E'_n\Delta P_n=P+R_n-PR_n+(1-P)E_n \Delta P_n\]

\[ R'_n = P+R_n-PR_n\]

\[ E'_n=(1-P)E_n\]

\(E'_n\) in XOR operation is

\[ R'_n + E'_n\Delta P_n=P + R_n-2PR_n+ (1-2P)E_n \Delta P_n \]

\[ R'_n = P + R_n-2PR_n\]

\[ E'_n = (1-2P)E_n\]

Next, we will show the method of backpropagation. If backpropagation is
from a logical operation, \(P^{r11}\) and \(P^{r00}\) are used
regardless of the operation. The formula is as follows.

\[ R'_n=1-P^{r00}+(P^{r11}+P^{r00}-1)Rn\]

\[ E'_n= (P^{r11}+P^{r00}-1)E_n\]

When backpropagating an AND operation, if the probability of the other
input is determined to be \(P_f\), the formula is as follows. The
derivation is complicated, so we will omit it. Although the quadratic
and higher terms of \(\Delta P_n\) are necessary, they are often
omitted.

\[ P'=P_fP_r = (R+E\Delta P_n)(R'+E'\Delta P_n)\]

\[ R'+E'\Delta P_n =\frac{P'}{R} - \frac{P'E}{R^2}\Delta P_n -\frac{P'E^2}{R^3}(\Delta P_n)^2 ...\]

\[ R' = \frac{P'}{R}\]

\[ E' = \frac{P'E}{R^2} \]

Backpropagation of OR operation is

\[ R' = \frac{1-P'}{1-R}\]

\[ E' = \frac{(1-P')E}{(1-R)^2} \]

Backpropagation of XOR operation is

\[ R' = \frac{P'-R}{1-2R}\]

\[ E' = \frac{(2P'-1)E}{(1-2R)^2} \]

By continuously applying the above formula to the network routes, the
propagation probability of the entire route can be calculated. This
propagation probability is considered to be equal to the observation
result \(P'_{total}\). All fluctuations of multiple links can be summed
up. Since the probability is less than 1, the product of probabilities
will definitely converge. Although the quadratic and higher terms of
\(\Delta P_n\) are necessary, they are often omitted.

\[ P'_{total} = R_{total}+ \sum _n E_{n total}\Delta P_n + \sum _m \sum _n E_{m total}E_{n total}\Delta P_m P_n ... \]

As a result, the final action probability \(E_{ntotal}\) obtained by
propagation is the coefficient of action for each link \(\Delta P_{n}\).

\[\sum_{n} E_{ntotal} \Delta P_{n} \approx P'_{total} - P_{total}\] 

\subsection{Derivation of the formula for weight value}\label{derivation-of-the-formula-for-weight-value-w_n}

The probability correction value \(\Delta P_n\) that is fed back to each
link that has passed through the propagation can be found by rewriting
and calculating all the probabilities of the propagation links in the
following form.

\[ P_n'=P_n + \Delta P_n\]

For the propagation probability \(P_n\) of link n, the fluctuation
occurrence probability \(t_n\) at which the propagation probability
\(P_n'\) is observed as an observation result follows a kind of binomial
distribution and is obtained by the following formula.

\[ t_n=\lim_{m \to \infty }\{\dbinom{m}{mP_n'} P^{mP_n'} (1-P_n)^{m(1-P_n')} \}^{-m}\]

Specifically, this formula uses m coin tosses with probability \(P_n\)
to find the probability that the sum of the results is \(mP~{'}_n\), and
takes the limit for m. The following combination formula is used.

\[ \dbinom{m}{mP_n'} =\frac{m!}{mP_n'! m(1-P_n')!} \]

Furthermore, by multiplying by the number of past observations of the
link \(N_n\), the probability that the result is \(mN_nP~{'}_n\) is
found. This is the probability of fluctuation occurring on a link where
probability \(P_n\) has been observed multiple times.

\[ T_n=\lim_{m \to \infty }\{\dbinom{mN_n}{mN_nP_n'} P^{mN_nP_n'} (1-P_n)^{mN_n(1-P_n')} \}^{-m}\]

First, the following ``Stirling's approximation formula'' is used to
calculate the value of the combination.

\[n!\sim {\sqrt {2\pi n}}\left({\frac {n}{e}}\right)^{n}\{1-\frac{1}{12n}+\frac{1}{288n^2}+... \}\]

The last constant other than 1 is omitted as an approximation, and the
following equation is used

\[n!\sim {\sqrt {2\pi n}}\left({\frac {n}{e}}\right)^{n} \]

Substituting into the combination equation using this,

\[\dbinom{mN_n}{mN_nP_n'}=\frac{mN_n!}{(mN_nP_n')! \{mN_n(1-P_n')\}!} \]

\[ \sim \frac{\sqrt{2\pi mN_n}}{\sqrt{2\pi mN_n 2\pi mN_n P_n'(1-P_n')}} \left(\frac{N_n}{e}\right)^{mN_n} )^{-mN_nP_n'} \left(\frac{N_n(1-P_n')}{e}\right)^{-mN_n(1-P_n')}\]

\[ = \frac{1}{\sqrt{2\pi mN_n P_n'(1-P_n')}}\frac{1}{P_n'^{mN_nP_n'} (1-P_n')^{mN_n(1-P_n')}}\]

Using the value of this combination, substitute \(T_n\).

\[ T_n=\lim_{m \to \infty } \{P_n^{mN_nP_n'}(1-P_n)^{mN_n(1-P_n')} P_n'^{-mN_nP_n'} (1-P_n')^{-mN_n(1-P_n')}\]

\[\{2\pi mN_n P_n'(1-P_n')\}^{-1/2} \}^{-m}\]

\[= P_n^{N_nP_n'}(1-P_n)^{N_n(1-P_n')} P_n'^{-N_nP_n'} (1-P_n')^{-N_n(1-P_n')} \lim_{m \to \infty } \{2\pi mN_n P_n'(1-P_n') \}^{-m/2} \]

This \(T_n\) is the probability that \(P_n'\) is observed in the link,
and we can find the variation of \(P_n\) to average this probability
variation over all links. Take both sides log and replace the left side
by \(\tau_n\). The last term converges in the limit of m, so we omit it.

\[ \tau_n = \log T_n = N_nP_n' \log P_n + N_n(1-P_n')\log(1-P_n) \]

\[- N_nP_n'\log P_n' -N_(1-P_n')\log(1-P_n') \]

\[ = N_n \{ P_n' \log P_n + (1-P_n')\log(1-P_n) - P_n'\log P_n' - (1-P_n')\log(1-P_n') \} \]

Differentiate this \(\tau_n\) equation with respect to \(P_n\).

\[ \frac{d\tau}{dP_n}=N_n \{ \frac{P_n'}{P_n}-\frac{1-P_n'}{1-P_n}\}\]

\[ = N_n \{ \frac{P_n'(1-P_n)-(1-P_n')P_n}{P_n(1-P_n)} \}\]

\[ = N_n \frac{P_n'-P_n}{P_n(1-P_n)} \]

\[ P_n'=P_n \pm \frac{P_n(1-P_n)}{N_n} \frac{d \tau_n}{d P_n} \]

This allows us to calculate the weight \(w_n\) of the fluctuation of
\(P_n\) with \(\tau\) as a parameter.

\[P_n + \Delta P_n\]

\[ \Delta P_n = \pm w_n \Delta \tau_n\]

\[
w_n = 
\begin{cases}
  \dfrac{P_n(1 - P_n)}{N_n} & \text{if } P_n > 0.5 \\
  -\dfrac{P_n(1 - P_n)}{N_n} & \text{if } P_n < 0.5
\end{cases}
\]

In this way, the weight $w_n$ of each link is determined by the known probability $P_n$ of the link and the number of observations $N_n$. 
The sign of the weight is unified in the direction that reduces the entropy of the link probability. 
Note that the sign of the weight is further reversed depending on the feedback target.

\subsection{Calculation of the probability correction value}\label{calculation-of-the-probability-correction-value-delta-p_n}

From here, we will explain the feedback distribution method to multiple
links. The observed total probability is \(P^{'}_{total}\). The
fluctuation \(\Delta P_n\) of each link is calculated so that it matches
this probability. \(E_n\) is the effect coefficient determined by the
propagation probability for link n, as mentioned above.

\[ P^{'}_{total}=P_{total}+ \Delta P_{total} = P_{total}+\sum_n E_n \Delta P_n + \sum _m \sum _n E_{m total}E_{n total}\Delta P_m P_n ...\]

Another constraint is to maximize the total fluctuation occurrence
probability \(T\).

\[ t_{all}=\prod_n t_n\]

Using \(\tau_n\), this formula becomes as follows.

\[T=\log t_{all} = \sum_n \log t_n = \sum_n \tau_n\]

The purpose of feedback is to determine the fluctuation value of \(P_n\)
for each link so that the total fluctuation occurrence probability \(T\)
(large tau) is maximized and the entropy of each link is minimized while
distributing the value of \(\tau_n\) as evenly as possible.

\[ \Delta P_{total}=\sum_{n} E_n w_n \frac{d \tau_n}{d P_n} \] 

\subsection{Feedback distribution and network replication}\label{feedback-distribution-and-network-replication}

\subsubsection{Equal distribution feedback}\label{equal-distribution-feedback}

When the amount of fluctuation to be fed back is small, equal
distribution is performed according to the weight of the link in order
to minimize the fluctuation of the entropy of the propagation
probability of each link. First, replace the fluctuation occurrence
probability \(\tau_n\) of each link n with a common parameter \(\tau\).

\[\Delta \tau = \Delta \tau_n\]

For each \(\Delta P_n\), the total fluctuation \(\Delta P_{total}\) is
distributed using the ratio of \(E_nw_n\).

\[\Delta P_{total} = \sum_n E_n w_n\Delta \tau\]

\[\Delta \tau = \frac{\Delta P_{total}}{ \sum_n E_n w_n} \]

Using the \(\Delta \tau\) calculated in this way, the total feedback
\(\Delta P_{total}\) can be distributed to the \(\Delta P_n\) of each
link.

\[ P_n' =P_n+\Delta P_n =P_n+w_n \Delta \tau\]

It is possible that \(P'_n\) can be greater than 1 or less than 0, so in
such cases \(P'_n\) is saturated to 1 or 0 and \(P_{total}\) is
recalculated each time. After that, the saturated link is removed and
\(\Delta \tau\) is calculated again.

However, there are cases where equal distribution is not possible, such
as when the sign of the probability is completely reversed from 1 to 0.
This is because the product of the fluctuations of each link cannot be
ignored. In actual feedback of binary logic, this case is more common.
Here is the procedure for doing so.

\subsubsection{Weight reversal feedback}\label{weight-reversal-feedback}

If the effect coefficient \(E_n\) of a link is close to 1, the overall
probability can be reversed by simply feeding back to that link. In this
case, in order to keep the propagation probability entropy of other
links at a minimum, one link with the smallest propagation probability
entropy and experience number N is selected and the weight is reversed.
Then, feedback is applied to completely reverse the overall probability.
Since there can be multiple links with an effect coefficient \(E_n\)
close to 1, there are multiple possibilities for link selection. For
each selection, the network is replicated and feedback is applied. The
priority is determined by the weight of the link. 

If the overall probability to be fed back is either 1 or 0, feedback is
completed by simply selecting and reversing one link. However, if the
overall probability to be fed back is an intermediate value between 1
and 0, after applying the reversal of one link, an equal distribution is
again performed to minimize the entropy of the other links. The purpose
of the feedback is to minimize the overall link entropy and to maximize
the fluctuation probabilistically, and if this purpose is met, the
weight is reversed.

\subsection{Link Propagation Probability Correction}\label{link-propagation-probability-correction}

This feedback-corrected probability does not become the probability that
the link itself will update. The larger the number of observations of
the link up to that point, \(N_n\), the smaller the link probability
correction amount will be. The value added to \(N_n\) is not always 1,
but is determined by the effectiveness coefficient \(E_n\) used in the
actual propagation and the number of observations of the other party
that is fed back, \(N_f\).

\[ N_n'=N_n+N_fE_n\]

As a result, the final update probability \(P''_n\) is as follows.

\[ {P''_n}^{N_n'}={P_n}^{N_n} {P'_n}^{N_fE_n}\]

Since there is not much difference between the multiplicative average
and the additive average, it can also be approximated by the following
formula.

\[ P''_n=\frac{N_nP_n +N_fE_nP'_n}{N_{n}'}\]

\(P_n\) is a mixture of links \(P^{11} and P^{00}\), and feedback is
applied to each of the links \(P^{11} and P^{00}\).

Based on the feedback results for the calculated probability
\(P^{11}_ n\), links where \(P^{11}_n\) is in the direction of
decreasing entropy, that is, where the probability approaches either 1
or 0, are considered to have received positive feedback. Conversely,
links where \(P^{11}_n\) is in the direction of increasing entropy, that
is, where the probability approaches 0.5, are considered to have
received negative feedback. It is assumed that negative feedback occurs
because the link has some invisible condition, and the condition is
searched for. Feedback is calculated similarly for \(P^{00}_n\).

When the number of link usages \(N_n\) is 0, the probability is
considered to be \(P_n=0.5\). Positive feedback alone can bring the
probability closer to 1 or 0, but it cannot become 1 or 0 itself.

This conditional link formation is implemented as part of the autonomous
generative mapping logic circuit algorithm described later.

\section{Associations}\label{associations}

\subsection{Associative Target Selection}\label{associative-target-selection}

An association is the process of predicting a causal relationship by
linking two collectively independent nodes with a map. To form an
association that may have a causal relationship, the success rate is
high if two nodes that fluctuated simultaneously are selected. Note that
simultaneous fluctuations include all fluctuations, such as time axes
and coordinates. The probability of simultaneous fluctuations, or in
other words, the strength of the association, can be quantitatively
determined based on the probability of fluctuations occurring at each
node.

The conditions for this are as follows.

\begin{enumerate}
\def\labelenumi{\arabic{enumi}.}
\item
  The probability entropy observed simultaneously is almost equal (in
  most cases, a definite probability of either 1 or 0).
\item
  The assumption vectors from the same origin are the same, or there is
  an inclusion relationship. Simultaneous fluctuations are considered to
  be cases in which the assumption vectors themselves are completely
  identical or have an inclusion relationship.
\end{enumerate}

Here, unrelated events can be accidentally activated at the same time
and become the subject of an association, but since the possibility of
further simultaneous activation of an association due to chance is low,
negative feedback results in an uncertain link.

\begin{figure}[ht]
  \centering
  \includegraphics[width=0.7\textwidth]{SOLE/DocumentImage/SOL_CreateAssociation.png}
  \caption{Create association}
  \label{fig:create_association}
\end{figure}

A node that is a candidate for association is the end point of the link
to which negative feedback has been applied. By applying negative
feedback, it can be considered that some condition has been added to the
link.

A node that changes at the same time as this negative feedback change is
considered to be possible for association. Furthermore, a change node
with a low probability of value change is considered to have a high
association accuracy, so it is selected with the highest priority.

An observation is an action of associating a node indicating the current
 space-time with a node indicating an observed value,
  and is basically generated each time by an external function.

\subsection{Association Formation and Addition of Conditions}\label{association-formation-and-addition-of-conditions}

\begin{figure}[ht]
  \centering
  \includegraphics[width=0.7\textwidth]{SOLE/DocumentImage/SOL_DifferenceofAssumptionVectors.png}
  \caption{Difference between assumption vectors}
  \label{fig:difference_between_assumption_vectors}
\end{figure}

Association requires the selection or generation of two nodes that are
conditions for performing the and from the map. These condition nodes
are generated from the difference of the assumption vectors. In the
above diagram, nodes F and C, which are the difference between AV1 and
AV2, are the conditions.

\subsection{Calculation of Propagation Probability and Number of Experiences}\label{calculation-of-propagation-probability-and-number-of-experiences}

Associative links are not definite links, and the number of association
experiences \(N\) can be determined from the probability of observed
simultaneous activation values.

\[N^{11}=log_2(P_Y)\]

The calculation method is as follows. In an unmeasured and uncertain
link, that is, an associative link with an assumed propagation
probability \(P=0.5\), \(P=1\) is actually observed \(N^{11}\) times,
and the past propagation probability \(P_Y\) of the link to the
associated object Y is measured as a result.

\[ 0.5^{N^{11}}= P_Y\]

In other words, \(P=0.5\) can be considered as a link observed \(N\)
times.

When X=1, Y=1 are observed, the association probability is \(P'=1\). As
a result, the initial values \hspace{0pt}\hspace{0pt}\(P^{11}, N^{11}\)
are as follows. When X=1, Y=0, \(P=0\).

\[N^{11}=log_2(P_Y)\]

\[P^{11}=\frac{0.5+ P'N^{11}}{1+N^{11}}\]

When X=0, Y=0 are observed, it only affects the \(P^{00}\) side. In this
case, the association probability is also \(P'=1\). As a result, the
initial state \(P^{00}, N^{00}\) is expressed by the following formula.
When X=0, Y=1, \(P'=0\).

\[N^{00}=log_2(1-P_Y)\]

\[P^{00}=\frac{0.5+ P'N^{00}}{1+N^{00}}\]

This calculation of the number of experiences means that the fluctuating
association between events that rarely occur is also highly certain of
the association.

\section{Autonomous Logic Generation Algorithm}\label{autonomous-logic-generation-algorithm-1}

This algorithm is a method for autonomously generating and modifying
nodes and links.

\subsection{``Conditional'' Link splitting and logical operation node insertion}\label{conditional-link-splitting-and-logical-operation-node-insertion-1}

\subsubsection{Feedback on associative links}\label{feedback-on-associative-links}

When an associative link is formed, the propagation probability is
uncertain, and the propagation probability is corrected by feedback. As
a result, while there are some links for which a definite propagation
probability of 1 or 0 is observed, there are many more indeterminate
links for which the propagation probability is an intermediate value. In
such cases, the indeterminate links for which both a propagation
probability value of 1 and 0 are observed are brought closer to a
definite probability of 1 or 0 by adding logical operation conditions.
This section explains the method of selecting the conditions for this
purpose.

\begin{figure}[ht]
  \centering
  \includegraphics[width=0.7\textwidth]{SOLE/DocumentImage/SOL_FeedbackToMapLinks.png}
  \caption{Feedback to an associative link}
  \label{fig:feedback_to_an_associative_link}
\end{figure}

Associations are formed from A and C to B. The associations are
connected by the Map1 and Map2 nodes, respectively. The propagation
probability \(P_{f11}\) of the two association links to B is corrected
to an uncertain probability by feedback for the same assumption vector.
To correct this propagation probability \(P_{f11}\) to a definite value
of 1 or 0, a condition is formed for the link of B. A logical operation
is inserted into the feedback side of the two uncertain links, and the
other becomes the input of the logical operation that becomes the
condition. A logical operation is not formed unless the link is
fluctuating for the same assumption vector.

\subsubsection{AND node}\label{and-node}

Conditioning is performed for link feedback to correct the link. Assume
that the link for which feedback is provided is the link from node A to
node B. In the following example, we assume that the propagation
probability of \(P^{11}\) has been reduced to 0 by feedback.

\[A \rightarrow B \quad
\begin{cases}
P^{11}=1 \rightarrow 0.8 \\
P^{00}=1
\end{cases}
\]

Use condition C corresponding to this feedback to add conditions. The
condition is AND.

\begin{figure}[ht]
  \centering
  \includegraphics[width=0.5\textwidth]{SOLE/DocumentImage/Fig_31.png}
  \caption{AND node formation}
  \label{fig:and_node_formation}
\end{figure}

\subsubsection{OR node}\label{or-node}

If feedback is made on \(P_{00}\), then the condition node C is ORed
together.

\[A \rightarrow B \quad
\begin{cases}
P^{11}=1 \\
P^{00}=1 \rightarrow 0.7
\end{cases}
\]

\begin{figure}[ht]
  \centering
  \includegraphics[width=0.5\textwidth]{SOLE/DocumentImage/Fig_32.png}
  \caption{OR node formation}
  \label{fig:or_node_formation}
\end{figure}

\subsubsection{XOR node}\label{xor-node}

If feedback is given to both \(P^{11}\) and \(P^{00}\), the link becomes
random and is essentially invalid. However, if feedback is given to both
\(P^{11}\) and \(P^{00}\) at the same time, the link becomes a candidate
for joining at the XOR node.

\[A \rightarrow B \quad
\begin{cases}
P^{11}=1 \rightarrow 0 \\
P^{00}=1 \rightarrow 0
\end{cases}
\]

\[C \rightarrow B \quad
\begin{cases}
P^{11}=1 \rightarrow 0 \\
P^{00}=1 \rightarrow 0
\end{cases}
\]

\begin{figure}[ht]
  \centering
  \includegraphics[width=0.5\textwidth]{SOLE/DocumentImage/Fig_33.png}
  \caption{XOR node formation}
  \label{fig:xor_node_formation}
\end{figure}

\subsubsection{NOT link}\label{not-link}

If the propagation probabilities of both \(P_{11}\) and \(P_{00}\) are
close to 0, it is a NOT link.

\[X \neg\rightarrow Y \quad
\begin{cases}
P^{11}=0.01 \\
P^{00}=0
\end{cases}
\]

\subsection{Link node activation from ``causal''
feedback}\label{link-node-activation-from-causal-feedback}

When feedback occurs to the following link, a link node is generated
from nodes A and B before and after the link.

\[ A \rightarrow B\]

\subsubsection{Logical operation equivalent to a link
node}\label{logical-operation-equivalent-to-a-link-node}

As a result of feedback to the link from A to B, link node L is
activated. The link from A to B is rewritten as a logical operation
using L. In other words, the true identity of link node L is a new input
node for the AND operation inserted into the link. If \(P_A=1\) and
\(P_B=1\) change to \(P_B=0\), logical operation AND\(\land\) is
inserted.

\[ B=A \land L\]

It becomes. If \(P_B=0\) and \(P_B=0\) change to \(P_B=1\), OR logical
operation \(LOR\) is inserted.

\[B=A \lor L\]

Furthermore, if feedback is observed for both \(P=1\) and \(P=0\), a
link node can be implemented using the XOR logic operation \(\oplus\).
For convenience, the link node and the XOR input are connected by a NOT
link.

\[B= A \oplus \neg L\]

If feedback occurs due to a collision of activation values, it is
propagated back to the link node.

\subsubsection{From determining the link propagation probability to
generating a link
node}\label{from-determining-the-link-propagation-probability-to-generating-a-link-node}

The probability of backpropagation to a link node using AND is given by
the following formula. \(P_A\) and \(P_B\) are the probabilities before
and after the link, and \(P_L\) is the probability of the link node.

\[ P_B=P_L P_A\]

The probability of propagation to a link node is given by the following
formula. Note that it is defined as anything other than \(P_A=0\).

\[ P_L=\frac{P_B}{P_A}\]

Note that unlike forward propagation, probability propagation to a link
node cannot propagate fluctuations as is.

The propagation set by the assumption vector is propagated by
substitution, with a concept similar to the backpropagation of a map.
The two propagation sets \(P=1\) and \(P=0\) propagated to A are
combined and propagated to the link node. As with the reverse
propagation of a map, the propagation probability to a link node when
\(P=0\) is indefinite, but the assumption vector propagated to the link
node is substituted for links with equal propagation probability.

Furthermore, the propagation probability to a link node is calculated
from the reverse propagation of XOR. First, the formula for the logical
operation of XOR is used.

\[ P_B=P_LP_A+(1-P_L)(1-P_A)\]

\[ P_L=\frac{P_A+P_B-1}{2P_A-1}\]

This formula is 1 if the probabilities of \(P_A\) and \(P_B\) are the
same, and 0 if they are complementary (the condition is that
\(P_A\neq 0.5\)).

Another feature of XOR link nodes is that it is strictly possible to
cancel out the assumption vector elements. If the complementary
propagation sets \(P=1\) and \(P=0\) originate from the same assumption
vector and the values \hspace{0pt}\hspace{0pt}of the assumption vector
elements are complementary, the assumption vector elements can be
integrated.

\subsection{``Substitution'' Propagation Connection between nodes in
a collectively contained
relationship}\label{substitution-propagation-connection-between-nodes-in-a-collectively-contained-relationship}

The term ``assignment'' used here means that when it is confirmed that
the hypothetical vectors are in an inclusive relationship with respect
to the activation values \hspace{0pt}\hspace{0pt}that have reached the
multiple mapping nodes, the entire activation values
\hspace{0pt}\hspace{0pt}are propagated between the mapping nodes.

\begin{figure}[ht]
  \centering
  \includegraphics[width=0.7\textwidth]{SOLE/DocumentImage/SOL_SubstituteintoFunction.png}
  \caption{Substitution between states using assumption vectors}
  \label{fig:substitution_between_states_using_assumption_vectors}
\end{figure}

In the figure, substitution is performed between multiple states. The
conditions for this substitution are explained below.

The localstate node contains a set of value A, but the rest of it is
indeterminate. For value a of the localstate node, the functionstate
node has set X which contains value A, so there is no inconsistency in
considering the functionstate node to contain the localstate node.
Therefore, the functionstate node contains the result node as a partial
state.

The result node is not present in the localstate node, but there is no
inconsistency in assuming that it exists in the undefined part of the
localstate node. This result can be added to the localstate node in the
next ``instantiation''. This is the assignment of a value to the state
by a function.

\subsection{``Instantiation'' Instance duplication of part of a
network using propagation
subsets}\label{instantiation-instance-duplication-of-part-of-a-network-using-propagation-subsets}

A part of a network is split vertically to take a subset, and a
simplified network is generated as an instance. The links generated by
instantiation are structurally the same as existing links, but the
selection of the output of many links can be omitted.

This instantiation is performed when the propagation probability for the
same propagation set is determined between the starting subset and the
ending subset, and the propagation set of the network for that route is
large by OR or the like. The entire route between the starting point and
the ending point is separated by the propagation set.

It can be used for purposes such as adding the result value of a
function to the input State.

\begin{figure}[ht]
  \centering
  \includegraphics[width=0.7\textwidth]{SOLE/DocumentImage/SOL_NetworkInstantiate.png}
  \caption{Partial instantiation of a network}
  \label{fig:partial_instantiation_of_a_network}
\end{figure}

\subsection{``Generalization'' Generalized duplication of part of a
network by propagation
subset}\label{generalization-generalized-duplication-of-part-of-a-network-by-propagation-subset}

A part of the established network is replicated as a more collectively
generalized network between the inclusion set of the origin and the
inclusion set of the destination. This is generalization. It is similar
to instantiation, but it is performed even when the propagation set of
the network is not completely observed. In other words, this is a
provisional network that is later corrected by feedback.

X is a generalized node that includes A. Y is a generalized node that
includes B. At this point, the relationship between X and Y is
undetermined. The path from X to Y is generalized when the propagation
from X to Y is determined using other elements C and D of X and Y, even
if A and B are activated with 0. At this time, the entire path between
the starting point X and the end point Y is separated. At that time, A,
which is a subset of X, and B, which is a subset of Y, are replaced with
variable V and W nodes, respectively.

It should be noted here that propagation is not determined for any
combination of the subsets A, C, and B, D inside X and Y, respectively.
For example, propagation from A to D is not established because it has
not been observed.

Generalized node X and generalized node Y may be OR nodes, but often use
Exclusive nodes, which mean mutually exclusive nodes.

\begin{figure}[ht]
  \centering
  \includegraphics[width=0.7\textwidth]{SOLE/DocumentImage/SOL_NetworkGeneralization.png}
  \caption{Partial generalization of the network}
  \label{fig:partial_generalization_of_the_network}
\end{figure}

\subsection{``Selection'' Selection control of multiple link
propagation}\label{selection-selection-control-of-multiple-link-propagation}

A large number of links may be connected as inputs and outputs of a
node. In particular, the number of links can be more than a thousand
when connecting functions. Therefore, in order to minimize search time,
it is necessary to selectively control link propagation and propagate
only to necessary links. The problem here is that the propagation
probability of these large number of links is originally deterministic,
and when link control is performed by logical operations, the
propagation probability fluctuates and becomes uncertain. To solve this
problem, a link selection node is generated separately from the control
by logical operations.

Link selection activates the link selection node. This activated link
selection node is linked to the fluctuating state of the rest of the
network by fluctuating association. As a result, it becomes possible to
select an appropriate link corresponding to the state of the network.

The link selection node can also be thought of as an association target
for a subset of the SOL network itself.

\subsection{``Convergence'' network
optimization}\label{convergence-network-optimization}

When positive feedback occurs, where two paths and probabilities match,
nodes and links in the paths of the network that are small in the
propagation set and do not have a large number of observations are
deleted. Used in SOL network optimization. It is the reverse action of
network duplication, and is performed on unused networks.


\begin{verbatim}
(A&(B|C)&D)|(A&(B|C)) = (A&(B|C))
\end{verbatim}

\subsection{Overall operation of SOL}\label{overall-operation-of-sol}

The entire SOL operates in the following steps using the algorithms described above.

  \begin{algorithm}
  \caption{SOL main loop}
  \begin{algorithmic}[1]
      \State Generate initial activation state
      \While{Overall loop}
          \State Select one activated node
          \If{Input shortage of logic node}
              \State Add assumption vector element to input
          \EndIf
          \State Probability collision determination using assumption vector
          \If{Probability collision detection}
              \State Execute feedback
              \If{Increase link entropy}
                \State Form condition
              \Else
                \State Form association
                \State Activate link node
              \EndIf
              \State Generate instance network
              \State Generate generalized network
          \EndIf
          \State Execute built-in function
          \For{Output links}
              \State Propagate to output link
              \State Execute logical operation
          \EndFor
      \EndWhile
  \end{algorithmic}
  \end{algorithm}

\clearpage

\section{Sequential circuit generation}\label{sequential-circuit-generation}

By using the states by mapping hierarchically, circuits between
generalized states are generated autonomously. Furthermore, by
generalizing the states with variables, etc., a function that can be
used universally is generated.

\subsection{Observation function}\label{observation-function}

The state observation function forms an association between the observed
current time \(T_n\) and the results A, B, and C observed at the same
time. The results observed at the same time at the current time are
further linked by mapping as causal relationships, and the logical
relationship between them is observed.

\begin{figure}[ht]
  \centering
  \includegraphics[width=0.7\textwidth]{SOLE/DocumentImage/SOL_Observation.png}
  \caption{Observation}
  \label{fig:observation}
\end{figure}

\subsection{State transition
generation}\label{state-transition-generation}

The state transition of a time series is formed from a mapping when time
is the difference. The mapping between A, B, C for time \(T_n\) and D
for time \(T_{n+1}\) is fed back to form a logic circuit between A, B,
C, and D.

\begin{figure}[ht]
  \centering
  \includegraphics[width=0.7\textwidth]{SOLE/DocumentImage/SOL_SequencialCircuitbyMap.png}
  \caption{Sequencial circuit by map}
  \label{fig:sequencial_circuit_by_map}
\end{figure}

\subsection{Function generation from generalized
nodes}\label{function-generation-from-generalized-nodes}

The mapping formed between observed nodes such as A, B, and result is
expanded and changed into an abstract node by feedback to \(P^{00}\) on
the link. For example, value A is ``generalized'' to become variable X1.
Similarly, value B is generalized to become variable X2. The association
between X1, X2, and out that is generalized in this way is expected to
be a function. Note that the relationship between individual A, B, and
result is expressed by a different mapping, and this different mapping
is selected by a function.

\begin{figure}[ht]
  \centering
  \includegraphics[width=0.7\textwidth]{SOLE/DocumentImage/SOL_FunctionMap.png}
  \caption{Function map}
  \label{fig:function_map}
\end{figure}

\subsection{Function execution}\label{function-execution}

The function formed by generalization assigns a state with the input
values. Since the function is considered to be propagation-aggregate
larger than the assigned state, the result value of the function is also
considered to be included in the input state. This is the application of
the function to the state. The applied result is then partially
replicated and added to the original state during instantiation.

When feedback occurs to a result node, the feedback is also recursively
applied to the function that produced the result.

\begin{figure}[ht]
  \centering
  \includegraphics[width=0.7\textwidth]{SOLE/DocumentImage/SOL_ApplyFunction.png}
  \caption{Apply function}
  \label{fig:apply_function}
\end{figure}

The result of the function can also be a link node indicating a match,
making it possible to implement a conditional test that converts a match
to a Boolean.

\subsection{Grouping functions}\label{grouping-functions}

It shows how to apply the generated functions continuously. In order to
continuously apply functions f and g, add a network that connects the
output of f and the input of g.

\begin{figure}[ht]
  \centering
  \includegraphics[width=0.7\textwidth]{SOLE/DocumentImage/SOL_FunctionGroup.png}
  \caption{Function group}
  \label{fig:function_group}
\end{figure}

Insert a condition and an AND node into the output of function f, and
connect it to the input of function g, so that the functions are
connected depending on the condition. By making this condition the
result of another function, it is possible to realize any combination of
functions, including recursion.

\subsection{Built-in functions and their
usage}\label{built-in-functions-and-their-usage}

A built-in function is defined as an external function that uses input
nodes and external observations to select and activate output nodes. The
action on the input is defined inside the built-in function. It can also
be used for input and output from outside SOL. The nodes used in the
built-in functions are Boolean values, but the built-in functions can
obtain the actual objects corresponding to the nodes. Specifically,
scalar values \hspace{0pt}\hspace{0pt}such as integers, vector values,
objects, etc. are associated and used by the built-in functions for
calculations. The results are generated as nodes corresponding to scalar
values, etc., and conditionally combined with the input state.

If you want to provide feedback to scalar values
\hspace{0pt}\hspace{0pt}with intermediate values, you need to use a
feedback method that is unrelated to probability. Vector values, etc.
cannot be managed by probability alone. Therefore, values
\hspace{0pt}\hspace{0pt}with intermediate values
\hspace{0pt}\hspace{0pt}are realized by preparing a mechanism other than
SOL's feedback. For example, generating functions such as Fourier series
and Gaussian noise, as well as addition and inner product operations
that integrate these, are also provided as built-in functions. The
selection and connection of these built-in functions can be realized by
SOL's mappings and functions.

In the example shown in the figure below, the real vector value of the
character Token is used to calculate the inner product using a built-in
function to generate a Result value and generate the corresponding node.

\begin{figure}[ht]
  \centering
  \includegraphics[width=0.7\textwidth]{SOLE/DocumentImage/SOL_EmbeddedFunction.png}
  \caption{Embedded function}
  \label{fig:embedded_function}
\end{figure}

\subsection{Comparison with Attention in Machine
Learning}\label{comparison-with-attention-in-machine-learning}

Attention{[}2{]}, the basis of the Transformer, which has been widely
used in recent years, is a mechanism that has produced remarkable
results in machine learning. This mechanism has application capabilities
that have not been achieved with previous logic circuits. However, this
mechanism is heuristic and does not have theoretical consistency.

\begin{figure}[ht]
  \centering
  \includegraphics[width=0.7\textwidth]{SOLE/DocumentImage/SOL_AttentionvsStateChange.png}
  \caption{Similarities between the Attention mechanism and SOL functions}
  \label{fig:similarities_between_the_attention_mechanism_and_sol_functions}
\end{figure}

Attention's KQVs are both N-dimensional vectors that correspond to
tokens such as characters. In contrast, SOL's states are unrestricted
hierarchical data structures.

The comparison of Attention's K and Q by inner product operation
corresponds to the comparison of the input conditions and input states
in SOL's functions. The partial state of SOL's input State is the Query
in Attention, and the comparison part of the SOL function becomes the
Key in Attention. As a result, the original State is substituted into
the function like the Value of Attention, and partial modification is
performed. The logical operations that are the output conditions of SOL
functions can be compared to the modification action on the Value of
Attention.

As described above, SOL's mapping and state functions can be compared to
the mechanism of Attention. If anything, SOL's functions can be
considered a more generalized and qualified version of the mechanism of
Attention.

\section{Conclusion}\label{conclusion}

To integrate and summarize the above contents, this self-organizing
logic has the following characteristics.

\begin{enumerate}
\def\labelenumi{\arabic{enumi}.}
\item
  Bidirectional logic realizes the concept of mapping by applying
  backpropagation to logical operations between sets. Mappings can be
  used to express relationships between arbitrary sets that do not
  overlap in space and time. This allows sequential circuits to be
  described using only logic circuits.
\item
  By setting the propagation probability of the binary pair \(P^{11}\),
  \(P^{00}\) in the connecting link between logical nodes and enabling
  bidirectional propagation of the link, bidirectional binary
  propagation is possible. It becomes a stochastic Bayesian network.
  Backpropagation is Bayes' theorem itself. Bidirectional propagation
  through the mapping allows exact propagation probability calculations
  between any two sets.
\item
  Collective probability propagation using assumption vectors strictly
  manages propagation subsets that pass through links and mappings using
  assumption vectors, and uses partial matches and differences of
  assumption vectors to generate information between any two related
  nodes. Set inclusion relationships can be calculated. When multiple
  probability propagations collide at the same node and both assumption
  vectors match, feedback is provided to both paths depending on whether
  the probabilities match or do not match. In particular, when feeding
  back an uncertain link to a definite value, a link node indicating a
  match is activated.
\item
  The bidirectional binomial stochastic Bayesian network automatically
  accurately feedback-corrects all link propagation probabilities by
  using a probability fluctuation distribution hierarchical feedback
  algorithm based on the past propagation probabilities of links and the
  number of observations.
\item
  Association probabilistically connects collectively exclusive nodes
  whose fluctuations are observed at the same time to create a mapping.
  The object of association is not only the observed value, but also the
  condition nodes and link nodes of logical operations that occur in
  feedback. This effect allows associations to be formed between any
  subset.
\item
  The probabilistic autonomous logic generation algorithm automatically
  generates deterministic digital logic with a probability of 100\% or
  0\% by adding logical operations and mappings according to the
  feedback results to the link probability connected by association. By
  using the propagation probability of the pair \(P^{11}\) and
  \(P^{00}\), accurate insertion of logical operations is possible. In
  this case, the network is partially replicated to add logical
  operations. In the case of fluctuations in \(P^{00}\), it is also
  possible to partially replicate the network and generalize it.
\item
  By forming associations for fluctuations in values
  \hspace{0pt}\hspace{0pt}over time and adding further logical
  operations for feedback to the associations, it is possible to
  generate generalized logic circuits between the input and output of
  memory, etc. Furthermore, generalized replication of the network
  autonomously forms general sequential circuits. The generalized
  sequential circuits are used many times to become functions.
\item
  By using mappings hierarchically, the hierarchical data structures and
  functions of general software can be expressed. By managing the
  containment relationship of the propagation sets between these
  mappings, we can check the match of parameters for a state with a data
  structure and add the result of the function to the state. To do this,
  we use ``assignment'' and ``instantiation'' between mappings defined
  in SOL.
\end{enumerate}

SOL is built only on basic mathematical principles such as probability,
sets, and mapping. It does not imitate biological neurons. In addition,
it hardly uses any non-rigorous heuristics such as the non-linear
activation function of neural networks. Therefore, it is closer to a
rigorous mathematical theory than general machine learning, and it is
almost certain that autonomous generation software is an extension of
SOL. Existing machine learning such as neural networks can even be
considered to be an approximate implementation of part of SOL.
Therefore, if we aim for accurate and safe machine learning, we have no
choice but to introduce the principles of SOL in some form.

Future challenges include demonstrating operation through
implementation, making the implementation more efficient, simplifying,
and parallelizing. In addition, whether SOL can comprehensively
autonomously generate logic circuits and software will be a future
research topic.

SOL has clear connections to mathematical logic, category theory, etc.,
so there is room for generalization of the theory.

\clearpage

\section{Glossary}\label{glossary}

\subsection*{Value node}\label{glossary-value-node}

A value node corresponds to a value that is the starting point of a
logical operation. It is activated by an assumption and serves as the
starting point of an assumption vector. Back propagation from logical
operations connected by links is also a type of assumption.

\subsection*{Logic node}\label{glossary-logic-node}

A logic operation is performed on the probabilities of multiple input
links, and the resulting probabilities are output to the output links.
While performing the logic operation, it also synthesizes the assumption
vector.

\subsection*{Joint node}\label{glossary-joint-node}

Outputs the inputs of multiple links as they are. No probability
operation is performed between inputs.

\subsection*{Function node}\label{glossary-function-node}

Performs an external function on the inputs of a link to activate the
result node.

\subsection*{Link}\label{glossary-link}

Combines nodes with a binomial set of propagation probabilities. Input
probabilities and return probabilities. The reverse probability is also
defined separately.

\subsection*{Link node}\label{glossary-link-node}

When the propagation probability of a link changes due to feedback, a
node corresponding to the link is generated and the comparison result of
the values before and after the link is propagated. Specifically, it
corresponds to the input backpropagation of a two-input AND node or XOR
node. If the probabilities before and after link propagation are equal,
1 is propagated to the link node. If the probabilities before and after
link propagation do not match, an uncertain value (a value between 0 and
1) is propagated to the link node.

\subsection*{Link selection node}\label{glossary-link-selection-node}

A node for selecting the propagation of a specific link. There is one
selection node for each link. By activating this node, the corresponding
link is preferentially selected.

\subsection*{SOL network}\label{glossary-sol-network}

It is a type of Bayesian network that consists of nodes and links and
propagates probabilities. Since links have propagation probability
weights, they are by definition networks rather than graphs. It is
characterized by the existence of logic operation nodes and the fact
that the propagation direction of logic operations is bidirectional.

\subsection*{Map}\label{glossary-map}

From a node that represents a map, multiple subset nodes are connected
via links and conditional AND nodes. This relationship between subset
nodes is considered a general map.

\subsection*{Assumption vector}\label{glossary-assumption-vector}

Assume the probability value of a specific link is either 0 or 1. It
becomes a propagation starting point by assuming an implicit input link
to the actual value node. A plurality of these assumptions come together
to form a binary vector. Logical operations integrate multiple
asusmption vectors. A combination node compares multiple asusmption
vectors and performs feedback if there is overlap in the assumption
vectors.

\subsection*{Propagation}\label{glossary-propagation}

Propagation of probability values through links. The propagation is
always the probability value from the origin, which is a value between 0
and 1. The assumption vector is propagated at the same time.

\subsection*{Reverse propagation}\label{glossary-reverse-propagation}

The action of retracing the input links of a logical operation. Some
logical operations perform the integration of assumption vectors.

\subsection*{Propagating set}\label{glossary-propagating-set}

A subset of the cross section through the nodes and links of a network,
determined by the assumption vectors alone. It is independent of the set
created by the logical operations of the nodes.

\subsection*{Activation value}\label{glossary-activation-value}

The activation state passing through the nodes and links of a network,
and is composed of a propagation set and a propagation probability.

\subsection*{Feedback}\label{glossary-feedback}

In bidirectional probability propagation in a network, when two paths
reach the same end point from the same starting point, the propagation
probabilities of the two paths and the assumed vector are compared. If
the hypothetical vectors overlap and the propagation probabilities are
not the same, feedback is used to correct the propagation probabilities
of the two paths so that they are the same. At that time, feedback
correction is given preferentially to uncertain links with a small
number of observations.

\subsection*{Effect coefficient}\label{glossary-effect-coefficient}

This shows how link fluctuations affect the results of link propagation.
Specifically, when the propagation probability of a link changes from 0
to 1, if the probability of the propagation destination in the network
definitely changes to 1, the effect coefficient is 1. On the other hand,
even if a link changes from 0 to 1, if it has passed through an OR
operation along the way, the probability at the end will not change. In
this case, the effect coefficient is 0.

\subsection*{Weight}\label{glossary-weight}

This is a value that indicates how much the variation of an individual
link affects the result of link propagation. If the link propagation
probability (value of \(P^{11}\) or \(P^{00}\)) is close to 0 or 1, the
weight is close to 0, and if the propagation probability is close to
0.5, the weight is maximum. The weight decreases as the number of link
feedback increases.

\subsection*{Association between
fluctuations}\label{glossary-association-between-fluctuations}

Association between nodes whose observed values fluctuate at the same
time is made by a mapping. Simultaneous fluctuations means that the
probability values of the same element of the assumption vector were
observed before and after the fluctuation, respectively.

\subsection*{Substitution}\label{glossary-substitution}

An action that virtually treats multiple nodes reached by the same
assumption vector as having an inclusive relationship. Even if only a
partial match, such as parameters, is confirmed, the entire set of nodes
is considered to be related if there are no contradictions.

\subsection*{Instantiation}\label{glossary-instantiation}

Replicate a part of the network as an instance. Since only the part of
the propagation set that has been determined is extracted, some of the
logical operations in the middle are omitted.

\subsection*{Generalization}\label{glossary-generalization}

Generalize and replicate a part of the network. A network with a larger
set of nodes than the observed nodes is generated.The nodes with larger
propagation sets are still only partially observed and thus
probabilistically uncertain at this point.

\subsection*{Embedded function}\label{glossary-embedded-function}

Provided to perform inputs and outputs to and from the outside of SOL.
It starts by propagating probabilities to inputs, and then uses
real-valued nodes corresponding to the inputs to propagate probabilities
to real-valued nodes corresponding to the results. Numerical
calculations and other operations are also performed by this embedded
function.


\begin{thebibliography}{99}

\bibitem{russell2010ai}
Stuart Russell and Peter Norvig,
\textit{Artificial Intelligence: A Modern Approach},
Prentice Hall, 2010.

\bibitem{vaswani2017attention}
Ashish Vaswani, Noam Shazeer, Niki Parmar, Jakob Uszkoreit, Llion Jones, Aidan N. Gomez, Lukasz Kaiser, and Illia Polosukhin,
\textit{Attention Is All You Need},
In *Advances in Neural Information Processing Systems (NeurIPS)*, 2017.

\end{thebibliography}

\end{document}